\section{Notation and Vocabulary}
\label{lexical considerations}
This report uses a slight variation of the commonly-used {\em Extended Backus 
Naur Form} (EBNF) to express the syntax of \emd{}.
Terminal symbols in \emd{} (i.e. symbols in its vocabulary)
are shown in the syntax descriptions in typewriter font as {\tt ,} or
{\tt "}, or in bold font for reserved words like
\kw{loop}.  
Non-terminal symbols are denoted by italicized English words
that intuitively illustrate the meaning of the syntactic constructs.
In EBNF, alternatives are indicated by $|$:
\begin{quote}A $|$ B\end{quote}
means choosing either $A$ or $B$; optional elements are shown using
square brackets [ ]:
\begin{quote}\opt{A}\end{quote}

means either zero or one $A$; and (possibly empty) sequences by
braces \{ \}:
\begin{quote}\oseq{A}\end{quote}

means zero or more repetitions of $A$.

Emerald is case insensitive---the case of input letters is significant only
in character literals and string literals.

\subsection{Literals}
Literal objects in \emd{} are divided into the following categories:

\begin{description}
\def\descriptionmargin{.08\linewidth}
\item[Numeric]~\\
\label{numeric literals}
\begin{quote}\it\begin{tabular}{lcl}
numericLiteral 	& $::=$  & \terminal{0x} \oseq{hexdigit} \\
		& $|$    & \terminal{0}  \oseq{octdigit} \\
		& $|$    & \seq{digit} \terminal{.} \oseq{digit} \\
		& $|$    & \seq{digit} \\
digit           & $::=$  & \terminal{0} $|$ \terminal{1} $|$ \terminal{2} $|$ \terminal{3} $|$ \terminal{4} $|$ \terminal{5} $|$ \terminal{6} $|$ \terminal{7} $|$ \terminal{8} $|$ \terminal{9} \\
octdigit           & $::=$  & \terminal{0} $|$ \terminal{1} $|$ \terminal{2} $|$ \terminal{3} $|$ \terminal{4} $|$ \terminal{5} $|$ \terminal{6} $|$ \terminal{7} \\
hexdigit           & $::=$  & digit $|$ \terminal{a} $|$ \terminal{b} $|$ \terminal{c} $|$ \terminal{d} $|$ \terminal{e} $|$ \terminal{f}
\end{tabular}\end{quote}
Numeric literals without a decimal point (\terminal{.}) denote objects of
the predefined type \tn{Integer}; those with decimal points
denote objects of the predefined type \tn{Real}.  
Literals beginning with \terminal{0x} are
interpreted in hexadecimal;
literals beginning with
\terminal{0} are interpreted in octal, 
For example, 12, 014, and 0xc
are \tn{Integer} literals representing the decimal number 12, and 2.1 and
215.45 are \tn{Real} literals.

\item[Booleans]~\\
\label{Boolean literals}
\begin{quote}\it\begin{tabular}{lcl}
booleanLiteral 	& $::=$  & \kw{true} $|$ \kw{false}
\end{tabular} \end{quote} 
The reserved words \kw{true} and \kw{false} refer to the two objects of
builtin type \tn{Boolean}.

\item[Nil]~\\
\label{Nil literals}
\begin{quote}\it\begin{tabular}{lcl}
nilLiteral 	& $::=$  & \kw{nil}
\end{tabular} \end{quote} 
The reserved word \kw{nil} refers to the distinguished nil object, whose
type is \tn{None}.

\item[Characters and Strings]~\\
\label{character literals}
\newcommand{\charlit}[1]{\terminal{'}\,#1\,\terminal{'}}

\begin{quote}\it\begin{tabular}{lcl}
characterLiteral &$::=$& \charlit{ccharacter} \\[1ex]
ccharacter & $::=$&	AnyCharacterExceptBackSlash  \\
& $|$ & scharacter\\[1ex]
stringLiteral &$::=$& \terminal{"} \{ scharacter \} \terminal{"}\\[1ex]
scharacter & $::=$&	AnyCharacterExceptDoubleQuoteOrBackSlash  \\
& $|$ & \terminal{\mybackslash} anyCharacterExceptUpArrow   \\
& $|$ & \terminal{\mybackslash}\terminal{\myuparrow}anyCharacter \\
& $|$ & \terminal{\mybackslash}oneTwoOrThreeOctalDigits
\end{tabular}\end{quote}
A character literal denotes an object of builtin type \tn{Character} and
consists of a single character written within single quotes.
The character \mybackslash{} permits the introduction of
escape sequences for the entry of special characters.
\mybackslash\mybackslash{} generates a single \mybackslash{} character,
\mybackslash\myuparrow{}C where C is any character
generates a control character in an implementation-defined 
manner\footnote{
In ASCII implementations, \mybackslash\myuparrow{}C
generates the ascii character formed by turning off the upper 2 bits 
in the character code for C.  Thus, \mybackslash\myuparrow{}J
is the newline character, and \mybackslash\myuparrow{}@ is the
null character.  The exception is the delete character, (octal 177) which is
generated by the sequence \mybackslash\myuparrow{}?.}.  Standard escape
sequences as in C (\mybackslash{}n, \mybackslash{}t,
etc.) are permitted, including one, two, or three octal digits
following a \mybackslash{} which represents a character by giving
its numerical (octal) equivalent.
\terminal{\mybackslash} followed by any other character stands for that
character.

Examples of characters are \charlit{A}, \charlit{r},
\charlit{\mybackslash\myuparrow{}C},  
\charlit{\mybackslash\mybackslash}, 
\charlit{\mybackslash\myuparrow{}?},
\charlit{\mybackslash\myuparrow{}J}, \charlit{\mybackslash{}n} and
\charlit{\mybackslash{}012};
the last three examples equivalently denote the newline character.

\label{string literals}
A string literal denotes an object of the builtin type \tn{String} and
consists of a possibly empty sequences of characters enclosed in double
quotes, using the same escape conventions as character literals.
Examples of strings are {\tt "Emerald City"},
{\tt "The \mybackslash{}"Evergreen\mybackslash{}" State"}, and {\tt ""}.

\item[Vectors]~\\
\label{vector literals}
\begin{quote}\it\begin{tabular}{lcl}
vectorLiteral &$::=$&
    \terminal{\lc} \sseq{expression}{\terminal{,}} \opt{\terminal{:} typeExpression}
    \terminal{\rc} 
\end{tabular}\end{quote}
A vector literal is a sequence of expressions enclosed in curly braces,
representing immutable (read-only) vectors.  The type of the expression is
{\it ImmutableVector.of[t]}, where t is either:
\begin{itemize}
  \item{} the type expression (if present), otherwise
  \item{} the syntactic type of the elements, if they are all the same,
  otherwise
  \item{} \tn{Any}
\end{itemize}

Examples of vector literals are \terminal{\lc{}1, 3, 5\rc} (with type {\it
ImmutableVector.of[Integer]}),
\terminal{\lc{}1, 3, 5 : \tn{Any}\rc} with type {\it ImmutableVector.of[Any]})
,
and \terminal{\lc{} 1, 'a', \kw{true}\rc} (with type {\it ImmutableVector.of[Any]}).

\item[Objects]~\\
Emerald objects are created using object literals (also known as object
constructors) or one of the syntactic extensions which translate to an
object constructor, which include class constructors, record constructors
and enumeration constructors.  These are discussed in Section \ref{objects}.
\item[Types]~\\
Emerald types are created using type constructors, which are described in 
\label{type literals}
Section \ref{type constructors}.
\end{description}

\subsection{Identifiers}
An \emd{} identifier is a non-empty sequence of letters, digits and the
underscore character ``{\myunderscore}'', beginning with a letter
or the underscore character. Identifiers are case-insensitive and
significant up to 64 characters in length.
Identifiers are used as reserved words, 
constant names, variable names, operation
names (cf. Section~\ref{operators}),
parameter names, and local names of objects.

\subsection{Reserved Identifiers}
\label{reserved words}\label{reserved identifiers}
Reserved identifiers are identifiers that have been reserved for special
use and may not be used otherwise as identifiers.
Reserved identifiers are further subdivided into keywords
and literals. 
\subsection*{Literals}
The reserved literal identifiers are:
\setlength{\boxlength}{1.39in}
\newcommand{\bkw}[1]{\blbox{\kw{#1}}}
\begin{quote}\sloppy
\bkw{false}
\bkw{nil}
\bkw{self}
\bkw{true}
\end{quote}

\subsection*{Keywords}
\label{keywords}
\emd{} keywords are used to delimit language
constructs; for example, the keywords \kw{loop} and \kw{end}~\kw{loop} are
used to enclose a loop body. 

The reserved keywords are:
\begin{quote}\sloppy
\bkw{all}
\bkw{and}
\bkw{as}
\bkw{assert}
\bkw{at}
\bkw{attached}
\bkw{awaiting}
\bkw{begin}
\bkw{builtin}
\bkw{by}
\bkw{checkpoint}
\bkw{class}
\bkw{codeof}
\bkw{confirm}
\bkw{const}
\bkw{else}
\bkw{elseif}
\bkw{end}
\bkw{enumeration}
\bkw{exit}
\bkw{export}
\bkw{external}
\bkw{failure}
\bkw{field}
\bkw{fix}
\bkw{for}
\bkw{forall}
\bkw{from}
\bkw{function}
\bkw{if}
\bkw{immutable}
\bkw{initially}
\bkw{isfixed}
\bkw{islocal}
\bkw{locate}
\bkw{loop}
\bkw{monitor}
\bkw{move}
\bkw{nameof}
\bkw{new}
\bkw{object}
\bkw{op}
\bkw{operation}
\bkw{or}
\bkw{primitive}
\bkw{process}
\bkw{record}
\bkw{recovery}
\bkw{refix}
\bkw{restrict}
\bkw{return}
\bkw{returnandfail}
\bkw{signal}
\bkw{syntactictypeof}
\bkw{suchthat}
\bkw{then}
\bkw{to}
\bkw{typeobject}
\bkw{typeof}
\bkw{unavailable}
\bkw{unfix}
\bkw{var}
\bkw{view}
\bkw{visit}
\bkw{wait}
\bkw{when}
\bkw{where}
\bkw{while}
\par
\end{quote}

\subsection{Operators}
\label{operators}
\setlength{\boxlength}{0.75in}
\begin{quote}\it\begin{tabular}{lcl}
operatorCharacter & $::=$ & \bcbox{!} $|$ \bcbox{\char '043} $|$ \bcbox{\char '046} $|$ \bcbox{*} \\
& $|$ & \bcbox{+} $|$ \bcbox{-} $|$ \bcbox{/} $|$ \bcbox{<} \\
& $|$ & \bcbox{=} $|$ \bcbox{>} $|$ \bcbox{?} $|$ \bcbox{@} \\
& $|$ & \bcbox{\myuparrow} $|$ \bcbox{|} $|$ \bcbox{\mytilde{}} \\
operator & $::=$ & \seq{operatorCharacter}
\end{tabular}\end{quote}
An operator is a non-empty sequence of operator characters.
Operators are used as punctuation and as operation names.  

\subsection*{Reserved Operators}
\label{reserved operators}
Reserved operators are operators that have been reserved for special
use and may not be used otherwise as operators or operation names.
Reserved operators are further subdivided into expression operators and
punctuation.

\subsection*{Expression operators}
Expression operators are used within expressions.  The reserved
expression operators are\footnote{The symbols typeset as \assign, \returns,
\conforms{}, and \matches{} are typed using 
 {\tt <-}, 
{\tt ->}, {\tt *>}, and {\tt *>} respectively.  The potential ambiguity
caused by using the same symbol for conforms and matches is resolved by
context.}:
\begin{quote}
\bcbox{\conforms} \bcbox{==} \bcbox{!==} \bcbox{\matches}
\end{quote}

These operators are described in Section \ref{expressions}.

\subsection*{Punctuation operators}
Punctuation operators are used to delimit language constructs.  The reserved
punctuation operators are:
\begin{quote}
\bcbox{\assign} \bcbox{\returns}
\end{quote}


\subsection{Separators}
Separators are sequences consisting of only spaces, tabs, and newlines; they
are used to separate consecutive language tokens.
Consecutive identifiers, operators  and/or numeric literals must be separated 
by at least one separator.

\subsection{Comments}
Comments in \emd{} are line-oriented. A comment starts on any
line with the comment
delimiter, {\tt \char '045}, and terminates at the end of the same line. The comment
delimiter is ignored within string and character literals.
A comment is lexically equivalent to a separator.
