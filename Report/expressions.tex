\section{Expressions}
\label{expressions}
Expressions are \emd{} constructs that denote objects.

\subsection{Literals and Identifier Expressions}
\begin{quote}\it\begin{tabular}{lcl}
expression &$::=$& literal \\
    & $|$ & constantIdentifier \\
    & $|$ & variableIdentifier
\end{tabular}\end{quote}
A literal expression directly denotes an object.
A constant identifier names the object it was initialized to while a variable
identifier names the object most recently bound to it.

\subsection{Operator Expressions}
Before examining \emd{} expressions that involve operators, we define the
precedence of the operators used. In Table~\ref{precedences},
the operators are ordered by increasing
precedence.  Operators of the same
precedence level are evaluated from left to right.

\begin{table}
\begin{center}
\begin{tabular}{||c|l|l||}
    \hline
     Precedence & Operator  & Operation                         \\
     Level      &           &                                   \\
    \hline
    \hline
     1      & \kw{view}-\kw{as} & Narrow or widen object's type    \\
            & \kw{restrict}-\kw{to} & Permanently narrow or widen object's type    \\
     2      & \myvertslash{}    & Logical or                    \\
            & \kw{or}           & Logical conditional (short-circuit) or  \\
     3      & \&                & Logical and                   \\
            & \kw{and}          & Logical conditional (short-circuit) and \\
     4      & !                 & Logical negation              \\
     5      & ==, !==           & Object identity and distinction \\
            & \conforms         & Type conformity           \\
            & $=,\;!=,\;<,\;<=,\;>=,\;>$& Relational operators   \\
     6      & $+$,$-$           & Additive operators              \\
     7      & *, /              & Multiplicative operators        \\
            & \#                & Modulus                         \\
            & User-defined      &             \\
     8 	    & $-$, \mytilde{}        & Arithmetic negation		\\
            & \kw{isfixed}      & Checks if object is fixed at node    \\
            & \kw{islocal}      & Checks if object is on the local node    \\
            & \kw{locate}       & Finds a possible location of the operand  \\
            & \kw{awaiting}     & Processes waiting on condition \\
            & \kw{codeof}       & Concrete type (implementation) of an object \\
            & \kw{nameof}       & Name of an object \\
            & \kw{typeof}       & Type of an object \\
            & \kw{syntactictypeof}       & Compile time type of an expression\\
            & \kw{welcome}      & Obtains a reference to an incoming object \\
            &                   & conforming to the operand\\
    \hline
\end{tabular}
\end{center}
    \caption{Precedence of Emerald Operators}
    \label{precedences}
\end{table}

\subsection{Reserved Operators}
The reserved operators have meanings defined by the language, and may not be
redefined.

\begin{quote}\it\begin{tabular}{lcl}
expression &$::=$& \sseq{expression}{reservedop} \\
reservedop &$::=$& \bcbox{$==$} $|$ \bcbox{$!==$} $|$ \bcbox{\conforms} $|$
	      \bcbox{\kw{or}} $|$ \bcbox{\kw{and}}
\end{tabular}\end{quote}
$==$ evaluates to \kw{true} if the two
expressions denote the same object; $!==$ is its opposite.
The \conforms{}  operator
compares its two operands (each of type \tn{type})
for conformity,
i.e., it evaluates to \kw{true} if the left operand conforms to
the right operand (cf. Section~\ref{conformity}).

The operator \kw{and} is a {\em conditional and} and evaluates as
follows: if the left operand evaluates to
\kw{false}, the result is \kw{false};
otherwise, the result is the value of the right expression. The
operator \kw{or} is a {\em conditional or} and evaluates as follows:
if the left operand evaluates to \kw{true}, the result is \kw{true};
otherwise, the result is the value of the right expression.

\begin{quote}\it\begin{tabular}{lcl}
expression &$::=$& \kw{locate} expression \\
& $|$ & \kw{isfixed} expression \\
& $|$ & \kw{islocal} expression \\
& $|$ & \kw{awaiting} expression \\
& $|$ & \kw{codeof} expression \\
& $|$ & \kw{nameof} expression \\
& $|$ & \kw{typeof} expression \\
& $|$ & \kw{syntactictypeof} expression \\
& $|$ & \kw{welcome} expression
\end{tabular}\end{quote}
\kw{isfixed} {\it e\/} evaluates to \kw{true} if {\it e\/}
is currently fixed at a site, otherwise it evaluates to \kw{false}.
\kw{islocal} {\it e\/} evaluates to true if {\it e\/} is currently on the
local node, and \kw{false} otherwise.
\kw{locate} {\it e\/} evaluates to an object (of type \tn{Node}) that
gives a location of the operand object during the execution of this
expression. This is
explained in detail in Section~\ref{locatics}.

The \kw{awaiting} operator takes as its operand an expression of
type \tn{Condition} and returns \kw{true} if at least one process
is suspended on the operand condition, and
\kw{false} otherwise (cf. Section~\ref{objects}).

\kw{codeof}, \kw{nameof}, and \kw{typeof}
\label{operator typeof}
return the concrete type object (a \tn{ConcreteType}), the name (a \tn{String}), or
the type (a \tn{Signature}) of any object (including \kw{nil}).
\kw{syntactictypeof} an expression returns the compile time type of an
expression (cf. Section~\ref{syntactictypeof}).  See
Appendix~\ref{builtin objects}
for a description of these builtin types.

\kw{welcome}
blocks until a welcomable object that conforms to the abstract type evaulated from the
following expression is moved onto the local node, and returns a reference to
that object. If the foreign object arrives from a discovered node, the node
graph of the local node is merged with the node graph of the object's origin.

\subsection{Invocations}
\begin{quote}\it\begin{tabular}{lcl}
expression & ::= & invocation
\end{tabular}\end{quote}
Any invocation which returns exactly one object may be used as an
expression.  Invocations are discussed in Section~\ref{operations}.

\subsection{Other Operators}
All other operators are translated into object invocations.  Each occurrence
of a unary operator is
translated into an invocation of the operand with the invocation name being
the name of the operator and with no arguments.  Each occurrence of a binary
operator is translated into an invocation of the left operand with the
invocation name being the name of the operator and with a single argument
which is the right operand.  For example: {$!${\it e}} is translated as {{\it
e\/}.$!$}, and {{\it a\/} $+$ {\it b\/}} is translated as {{\it {\it a\/}.$+$\LB{}b\/}\RB}.


\subsection{Field selection}
\emd{} supports syntactic sugar to facilitate accessing the data components
of objects.  Conventionally, a visible data component named {\it f}
will have an operation named {\it getF} which returns the value of the
component and an operation {\it setF} which modifies the value of the
component.  Two syntactic forms support this convention by making the
invocation of get and set operations more convenient.

\begin{quote}\it\begin{tabular}{lcl}
  fieldSelection & $::=$ & expression \terminal{\$} identifier \\
\end{tabular}\end{quote}

In an expression context (as an r-value) {{\it a}\${\it f}} is translated as
{{\it a\/}.{\it getF}} while in an assignment context (as an l-value) {{\it
a}\${\it f} \assign{} c}  is translated as {a.setF\/\LB{}c\/\RB{}}.  The get operation
always takes no arguments and returns one result, while the set operation
always takes one argument and returns no results.
\subsection{Subscripts}
By convention, subscriptable values like \tn{Vectors} and \tn{Arrays} will
have an operation {\it getElement} to retrieve values and an operation {\it
setElement} to store values.  This convention is supported by the
subscripting expression syntax:

\begin{quote}\it\begin{tabular}{lcl}
  subscript      & $::=$ & expression \terminal{[}
    \sseq{expression}{\terminal{,}} \terminal{]}
\end{tabular}\end{quote}

The variable number of expressions inside the square brackets may be used to
support subscriptable values of higher dimension, or to select ranges of
values.  Such interpretation is up to the implementor of the subscripted
object.

In an expression context (as an r-value) {{\it a\/}\/\LB{\it b, c, d}\/\RB}
is translated as {{\it a\/}.{\it getElement}\LB{}{\it b, c, d}\RB}, while in
an assignment context (as an l-value) {a\/\LB{}b, c, d\/\RB{} \assign{} e}
is translated as {a\/.setElement\LB{}b, c, d, e\RB}.

\subsection{Type widening and narrowing}
\label{view expression}
\begin{quote}\it\begin{tabular}{lcl}
expression &$::=$& \kw{view} expression  \kw{as}  typeExpression \\
 &$|$& \kw{restrict} expression  \kw{to}  typeExpression
\end{tabular}\end{quote}

The \kw{view} expression permits an
object to be regarded as being of a different type, subject to
the restriction that no object expression be viewed as a type it does
not conform to. In other words, this expression permits the user to narrow
or widen the type of an object.  The restrict expression is similar, except
that it is later impossible to widen the type of the resulting object
reference to a type wider than that denoted by {\it typeExpression}.
