\section{Compiling Emerald Programs}
\label{compiler appendix}

This appendix presents a sample \emd{} program, and shows how
Emerald programs are compiled and executed.  
Examples of the Emerald style of programming can be found
in \cite{Raj90spe}.

The latest implementation of Emerald uses an interpreter to increase
portability.  The compiler is written in Emerald, and generates executable
files containing byte codes which are then interpreted.  The two commands of
interest are:
\begin{description}
  \item[emc]{}~\\
    The emerald compiler is interactive.  The primary commands of interest
    to casual Emerald users are:
    \begin{description}
      \item[{\bf load} filename]{} ~\\
	Define the environment in which this compilation is to be executed.
	Environments are stored in Unix files in the working directory.  
	Exported identifiers are made known in the environment
	in order to facilitate separate compilation.  A compilation
	environment consists of two files:  {\it filename\/} and {\it
	filename\/}.idb.  Both will be created as needed if they do not
	already exist.
      \item[filename]{} ~\\
	Compile the given file (whose name should end in .m).  The
	executable is written to a file with the extension .x.
    \end{description}
  \item[emx]{} ~\\
    The emerald interpreter.  Typically invoked as:
    \begin{quote}
      emx filename.x
    \end{quote}
    Emx supports a large number of flags, the most useful of which are:
    \begin{description}
      \item[-i]{} ~\\
	Invoke the integrated Emerald debugger when errors occur.
      \item[-Tcall]{} ~\\
	Generate a call trace giving the calls and returns of all operations
	and functions.
      \item[-v]{} ~\\
	Generate summary statistics on program termination.
    \end{description}
\end{description}

The environment mechanism 
permits the sharing of object
definitions between compilations.  Objects (actually, identifiers) exported
from one compilation unit that are to be used in another must be exported to
the environment in order to be visible to later compilation units.
An Emerald compilation unit is defined as follows:

\begin{center}
\begin{tabular}{lcl}
compilation & $::=$   & \oseq{environmentExport $|$ constantDeclaration } \\
environmentExport
            & $::=$   & \kw{export} \sseq{identifier}{\terminal{,}} \\
\end{tabular}
\end{center}

For example, if an object $Directory$ is being defined in compilation, and
needs to be used by other objects (being compiled later), a statement:

\begin{quote}
  \kw{export}  \it Directory
\end{quote}

is needed. Other source code that uses the identifier {\it Directory} must
be compiled in the context of this export.

\subsection{A Sample Program}
\label{sample program}

This simple example involves an integer stack creator ({\it Stack\/}),
which is compiled first, and a test object ({\it Tester\/}), which is compiled
later and makes use of the $Stack$. The two programs are entered into
separate Unix files.

\subsubsection*{The file {\tt stack.m}}
\begin{figure}[tbp]
{\small\it\begin{minipage}{\textwidth}\begin{tabbing}
xxx\=xxx\=xxx\=xxx\=xxx\=xxx\=xxx\=xxx\=xxx\=xxx\=xxx\=xxx\=xxx\=\+\kill%
\cd{} Define an \tn{Integer} stack creation \kw{object}\\[1.0ex]{}%
\kw{export} Stack\\[1.0ex]{}%
\kw{const} Stack \assign{}\+\\*{}%
  \kw{immutable} \kw{object} Stack\+\\*{}%
    \kw{const} StackType \assign{}\+\\*{}%
      \kw{typeobject} StackType\+\\*{}%
        \kw{operation} Push\/\LB{}n\CO{} \tn{Integer}\/\RB{}\\*{}%
        \kw{operation} Pop \returns{} \/\LB{}n\CO{} \tn{Integer}\/\RB{}\\*{}%
        \kw{function} Empty \returns{} \/\LB{}result \CO{} \tn{Boolean}\/\RB{}\-\\*{}%
      \kw{end} StackType\-\\[1.0ex]{}%
    \kw{export} \kw{function} getSignature \returns{} \/\LB{}r \CO{} \tn{Signature}\/\RB{}\+\\*{}%
      r \assign{} StackType\-\\*{}%
    \kw{end} getSignature\\[1.0ex]{}%
    \kw{export} \kw{operation} create \returns{} \/\LB{}result \CO{} StackType\/\RB{}\+\\*{}%
      result \assign{} \+\\*{}%
        \kw{object} aStack\+\\*{}%
	  \kw{const} store \assign{} \tn{Array}.of\/\LB{}\tn{Integer}\/\RB{}.empty\\[1.0ex]{}%
	  \kw{export} \kw{operation} Push\/\LB{}n\CO{} \tn{Integer}\/\RB{}\+\\*{}%
	    store.addUpper\/\LB{}n\/\RB{}\-\\*{}%
	  \kw{end} Push\\[1.0ex]{}%
	  \kw{export} \kw{operation} Pop \returns{} \/\LB{}n\CO{} \tn{Integer}\/\RB{}\+\\*{}%
	    n \assign{} store.removeUpper\-\\*{}%
	  \kw{end} Pop\\[1.0ex]{}%
	  \kw{export} \kw{function} Empty \returns{} \/\LB{}result \CO{} \tn{Boolean}\/\RB{}\+\\*{}%
	    result \assign{} store.empty\-\\*{}%
	  \kw{end} Empty\-\\*{}%
        \kw{end} aStack\-\-\\*{}%
    \kw{end} create\-\-\\[1.0ex]{}%
\kw{end} Stack
\end{tabbing}\end{minipage}}

\caption{The file {\tt stack.m}}
\label{STACK PROG}
\end{figure}
This Emerald program file defines the stack-creator (see Figure~\ref{STACK
PROG}); once created, this object
accepts $create$ invocations and returns a new stack-like object
(conforming to $Stacktype$) on each such invocation. See
\cite{Hutchinson87,Raj90spe} for a better understanding of this
program.  Note that the name Stack is exported to the compilation
environment using the export directive.
    
\subsection*{The file {\tt tester.m}}
\begin{figure}
{\small\it\begin{minipage}{\textwidth}\begin{tabbing}
xxx\=xxx\=xxx\=xxx\=xxx\=xxx\=xxx\=xxx\=xxx\=xxx\=xxx\=xxx\=xxx\=\+\kill%
\cd{} This program first creates a stack named myStack \kw{by} invoking Stack.create.\\*{}%
\cd{} It pushes 4 integers into myStack, \kw{and} \kw{then} pops \kw{and} prints them.\\[1.0ex]{}%
\kw{const} Tester \assign{} \kw{object} Tester\+\\*{}%
  \kw{process}\+\\*{}%
    \kw{const} myStack\CO{} Stack \assign{} Stack.create\\[1.0ex]{}%
    \kw{for} i \CO{} \tn{Integer} \assign{} 0 \kw{while} i $<$ 4 \kw{by} i \assign{} i $+$ 1\+\\*{}%
      stdout.PutString\/\LB{}``Pushing " $|$$|$ i.asString $|$$|$ " \kw{on} my stack.\mybackslash{}n''\/\RB{}\\*{}%
      myStack.Push\/\LB{}i\/\RB{}\-\\*{}%
    \kw{end} \kw{for}\\[1.0ex]{}%
    stdout.PutString\/\LB{}``Printing in reverse order.\mybackslash{}n''\/\RB{}\\*{}%
    \kw{loop}\+\\*{}%
      \kw{var} x\CO{} \tn{Integer}\\*{}%
      \kw{exit} \kw{when} myStack.Empty\\*{}%
      x \assign{} myStack.Pop\\*{}%
      stdout.PutString\/\LB{}``Popped " $|$$|$ x.asString $|$$|$ " from my stack.\mybackslash{}n''\/\RB{}\-\\*{}%
    \kw{end} \kw{loop}\\*{}%
    stdout.close\\*{}%
    stdin.close\-\\*{}%
  \kw{end} \kw{process}\-\\*{}%
\kw{end} Tester
\end{tabbing}\end{minipage}}

\caption{The file {\tt tester.m}}
\label{TEST STACK PROG}
\end{figure}
This file (see~Figure~\ref{TEST STACK PROG}) contains a simple object that
can be used to 
test the $Stack$ object defined in Figure~\ref{STACK PROG}.
This object invokes the stack-creator object $Stack$
to create the new stack named $myStack$; the rest of the program is fairly
straight-forward. The name {\it Stack\/} in unbound in this example,
therefore this program must be compiled in a compilation environment that
includes the name {\it Stack\/}.
Also note the predefined identifier {\it stdin\/} and {\it stdout\/} which name
the (already opened) standard input and output streams respectively.

\subsection{Compiling the Program}
\label{compiling program}
Compiling the program requires the following steps.  We assume that the
source files are named stack.m and stacktest.m.  The following steps provide
input to the compiler at its ``Command:'' prompt, after it has been started
using the Unix command ``emc''.
\begin{enumerate}
  \item{} Direct the compiler to load the stack environment:  ``load
    stackenv''.
  \item{} Compile the stack creator: ``stack.m''.
  \item{} Compile the stack tester: ``stacktest.m''.
  \item{} Terminate the compiler ``quit''.
\end{enumerate}
\subsection{Executing the Program}
\label{executing program}
The previous steps have generated two emerald executable files, stack.x and
stacktest.x.  These must be executed together:
\begin{quote}
  emx stack.x stacktest.x
\end{quote}

On executing this command, we get the following output:

\begin{verbatim}
        Pushing 0 on my stack.
        Pushing 1 on my stack.
        Pushing 2 on my stack.
        Pushing 3 on my stack.
        Printing in reverse order.
        Popped 3 from my stack.
        Popped 2 from my stack.
        Popped 1 from my stack.
        Popped 0 from my stack.
\end{verbatim}
