\section{Debugging Emerald Programs}
\label{debugger appendix}

This appendix presents the interface to the Emerald debugger, edb.  Edb is
built into the interpreter and is enabled by invoking the interpreter with
the -i option, as either:
\begin{quote}
  emx -i broken.x
\end{quote}
or
\begin{quote}
  broken.x -i
\end{quote}

\subsection{Breakpoints}
The debugger currently does not support breakpoints, but it does allow one
to continue past a failed assertion.  Therefore the user may insert
breakpoints by:
\begin{enumerate}
  \item{} Inserting an assertion into the source program.  For a conditional
    breakpoint, assert the negation of the condition that you want to cause
    the program to stop.  For an unconditional breakpoint, assert false.
  \item{} Recompile the program.  Note that if you are compiling a large
    system, using ``unset perfile'' will cause all generated code to be
    added to a single file, named CP (for checkpoint), and later added code
    will overwrite previous code.
  \item{} Run the code.  When the breakpoint happens, you can poke around,
    and continue using the continue command.
\end{enumerate}

\subsection{Commands}
The debugger supports the following commands:
\begin{description}
  \item[quit]{}~
  \item[q]{}~\\
    Terminate both the debugger and the interpreter.
  \item[where]{}~\\
    Provide a brief stack backtrace of the current process.
  \item[dump]{}~\\
    Provide a complete stack backtrace of the current process.  This
    backtrace includes instance variable of the target of each invocation as
    well as parameters, results, and local variables in each invocation.
  \item[processes]{}~\\
    List all the processes in the current program.  For each, the location
    of the bottom (most recent) invocation is given.
  \item[process n]{}~\\
    Switch the debugger's attention to the stack of process number n.  The
    available processes are given using the ``processes'' command.
  \item[continue]{}~
  \item[cont]{}~
  \item[c]{}~\\
    Continue execution of the target program.  This will likely cause
    tremendous problems if the cause of the program's stop was anything
    other than a failed assertion.  Segmentation faults can be expected for
    continuing through other failures.
  \item[print e]{}~
  \item[p e]{}~\\
    Display the value of the expression e.  Expressions take the form of
    identifier\{.identifier\}.  Spaces are not permitted in expressions.
    The current call stack is searched for the first identifier, additional
    identifiers are taken to be instance variables with that object.  The
    debugger prints as much information as it can determine about the named
    object.  To look at the current object (the one that was invoked at the
    current stack level) use the name ``self''.
  \item[up]{}~\\
    Move up the call stack (towards older activations) one level.
  \item[down]{}~\\
    Move down the call stack (towards younger activations) one level.
  \item[look]{}~
  \item[info]{}~\\
    Print as much information as is available about the current activation.
\end{description}
