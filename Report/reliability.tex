\section{Location and Reliability}
\label{location and reliability}
\label{locatics}
\emd{} was developed primarily to facilitate the construction of distributed
application programs.  To be resilient to machine crashes, these programs
should be capable of detecting and recovering from such crashes. They should
also be able to control the location of component objects so that the
available nodes in the system are optimally exploited. This section
discusses the \emd{} location-related constructs.

\label{location}
There are two \emd{} concepts that concern location.  These correspond to
two desires that motivate application programmers to deal with location.  As
stated previously, invocation in \emd{} is location independent.  This means
that the location of an object need not be determined in order to invoke it.
There are however two considerations that we expect to motivate application
programmers to concern themselves with location: performance and
reliability/availability.
\subsubsection*{Performance}
Since remote invocation will necessarily be at least an order of magnitude
more expensive than local invocation, the placement of \emd{} objects may
seriously affect their performance.  In order to provide the programmer with
control over the placement of objects the move statement (see Section
\ref{move statement}) is provided.  In addition, the call-by-move
implementation strategy for arguments to invocations (see Section
\ref{call by move}) allows further optimizations.
\subsubsection*{Reliability and Availability}
Since an object may be moved at arbitrary times by any other object with a
reference to it, a more permanent binding between objects and locations is
often required.  In particular, in order to implement an available
replicated service, it is necessary to place the replicas on differing
machines and not allow them to move.  This allows the programmer to
guarantee that a single machine failure will not cause more than one of his
replicas to become unavailable.

In order to provide for this requirement, the fix and unfix statements
(see Sections \ref{fix statement} and \ref{unfix statement}) may be used.
An object, once fixed at a particular location, may not be moved from there.
Any attempt to do so will fail (see subsection \ref{failures}).
\subsection{Unavailable objects}
\label{unavailable objects}
Due to machine crashes or communication network failures, objects may be
temporarily or permanently unavailable.
\emd{} provides unavailable handlers to allow
programmers to detect such situations and attempt recovery.
\begin{quote}\it\begin{tabular}{lcl}
unavailableHandler & $::=$ & \kw{unavailable} \opt{\terminal{[}identifier \terminal{]}}\\
  &  &  \hspace*{0.5in}blockBody \\
  &  & \kw{end} \kw{unavailable}
\end{tabular}\end{quote}
An object is regarded as being unavailable when it cannot be located
at any available node following suitable system action \cite{Jul88tocs}. 
When an attempt is made to invoke an object which is unavailable, the
appropriate hander is located in a manner similar to that for failures (see
Section~\ref{failures}), the 
unavailable object is bound to the identifier declared in that unavailable
handler (if present), and the body of the handler is executed.  The type of
the identifier in the handler definition is \tn{Any}.

While invoking an object which is unavailable results in an unavailable
exception, the typeof, codeof, and nameof expressions will correctly
identify even
unavailable objects.

\subsection{Failures}
\label{failures}
Failures can result from a number of causes; these include
attempting to invoke a \kw{nil} reference, assertion failures, divide-by-zero
and subscript-range errors.
\begin{quote}\it\begin{tabular}{lcl}
failureHandler & $::=$ & \kw{failure} \\
  &  &  \hspace*{0.5in}blockBody \\
  &  & \kw{end} \kw{failure}
\end{tabular}\end{quote}
After a failure is detected, the following action is taken.
\begin{enumerate}
\item{} The appropriate failure handler to execute is found.  This handler is
the handler attached to the smallest block containing the statement.  Note
that failures are considered a ``superclass'' of unavailables, and so when
an unavailable exception has been raised, each block that may have a handler
is first searched for an unavailable handler, and then a failure handler,
and only if no handler can be found is the next larger enclosing block
searched.  Similarly when propagating an unavailable exception up the call
stack, each block is first searched for an unavailable handler and then for
a failure handler.
\item{} If the block body of a monitored operation,
the initially section, or the recovery section of an object fails,
then
the object is said to have failed.  Any subsequent invocation attempted on the
object will fail, and any invocations that have started but have not yet
completed also fail.
\item{} An unhandled failure in the block body of an operation is propagated
by causing the corresponding invocation to fail.
\item{} An unhandled failure in an initially section implies that
the object creation has failed; this is propagated by causing
the statement containing the object constructor expression to fail.
\item{} An unhandled failure in the block body of a recovery section cannot be
propagated because  its execution did not result from an invocation.
\item{} An unhandled failure in a process block body cannot be propagated.
The process is terminated, but the object itself does not fail.
\item{} When an object fails, no attempt is made to immediately track down
and fail all processes (including the one contained in the object) that have
threads of control that have passed through the object.  When these threads
of control return to the body of any operation inside the object, they
will then fail.
\end{enumerate}
