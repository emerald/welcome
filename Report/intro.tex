\section{Introduction}

The primary goal of \emd{}
\cite{Black86-oopsla,Black87-tse,Jul88tocs,Raj90spe} is to simplify distributed
programming through language support while providing acceptable
performance and flexibility both in local and distributed environments.
Emerald also demonstrates that the object-based model of programming can
be incorporated both elegantly and efficiently in distributed systems.
\emd{} draws heavily upon the experience gained from
Smalltalk \cite{Goldberg83}, the Argus Language and System
\cite{Liskov84-argus} and, in particular, the Eden system
\cite{Almes85,Black85-eden-experiences} and the Eden Programming
Language (EPL) \cite{Black85-epl}.

Featuring an object-oriented style of programming, \emd{} presents
a unified semantic view of objects appropriate for
private, local, data-only objects as well as shared,
remote, concurrently-executing objects. The nature of objects in \emd{}
is similar to that in Smalltalk \cite{Goldberg83}, i.e., all 
data items are objects with a uniform semantic model for operations
on them, but \emd{} does not have any notion of {\em class}.
\emd{} was explicitly
designed to support data abstraction: all
typing of objects is at an abstract level and does not depend on the 
implementation chosen. Abstract typing aids
in the dynamic construction of distributed programs by allowing any object in
a large, possibly distributed, program to be replaced by any other
type-consistent object.
Type consistency or {\em conformity} is an important
aspect of Emerald, and is discussed below.
Another advantage of treating types as first class objects is 
that it makes polymorphism inherent in \emd{}.

Recognizing {\em location} as an important attribute of an object in
distributed programs, \emd{} gives the programmer access to the location
of objects through primitives that permit the inspection and selection of
location. Alternatively, when desired, the location details may be left
to the reasonably-chosen system-defaults.
However, this recognition of the importance of location 
for distributed programming has its drawbacks, viz., the semantics of \emd{}
are complicated both because location is apparent and because systems may
be partially unavailable.

This report defines the Emerald programming language.
The Emerald
approach to programming is discussed in \cite{Raj90spe}, where several
examples of Emerald programs may be found.

