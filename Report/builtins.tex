\section{Built-in Objects}
\label{builtin objects}
This appendix defines the built-in objects.  These descriptions will mention
that an object is a type rather than explicitly refering to the object's
immutability and its getSignature operation.
\newcommand{\opd}[1]{\makebox[1cm]{#1}}

\subsection{Any}
\label{builtin Any}
Any is the type that requires no operations; every Emerald object has type
Any.

\subsection{Array}
\label{builtin Array}
Arrays implement expandable indexable storage.  The of operation on
Array takes a Type, and returns an array creator.  As arrays
can expand and shrink, common data types such as Stacks and Queues can
be implemented using Arrays:  Stacks use addUpper and removeUpper,
while Queues use addUpper and removeLower.

\newenvironment{desc}
               {\list{}{\itemsep 0pt \parsep 0pt \labelwidth 0pt \itemindent-\leftmargin\labelsep 0pt
                        \let\makelabel\desclabel}}
               {\endlist}
\newcommand*\desclabel[1]{\hspace\labelsep\it #1}

\noindent The object Array is immutable and has the following interface:
\begin{desc}
  \item[\kw{function} of\/\LB{}T \CO{} \tn{type}\/\RB{} \returns{}
  \/\LB{}aNewArrayCreatorType\/\RB{} \kw{forall} T]
\end{desc}

\noindent The object resulting from {\it \tn{Array}.of\/\LB{}T\/\RB{}} (with
type aNewArrayCreatorType)
is a creator as well as a type.  It is immutable and has the following
interface:

\begin{desc}
  \item[\kw{operation} empty \returns{} \/\LB{}aNewArrayType\/\RB{}]~\\
    Return a new empty \tn{Array}.
  \item[\kw{operation} literal\/\LB{}Sequence.of\/\LB{}T\/\RB\/\RB{} \returns{} \/\LB{}aNewArrayType\/\RB{}]~\\
    Return a new \tn{Array} initialized with all the elements
    from the given sequence.
  \item[\kw{operation} create\/\LB{}size \CO{} \tn{Integer}\/\RB{} \returns{} \/\LB{}aNewArrayType\/\RB{}]~\\
    Return a new \tn{Array} with size elements all initialized to \kw{nil}.
\end{desc}

\noindent Objects with type {\it \tn{Array}.of\/\LB{}T\/\RB{}} have the
following interface, named aNewArrayType:

\begin{desc}
  \item[\kw{function}  getElement\/\LB{}index \CO{} \tn{Integer}\/\RB{} \returns{} \/\LB{}T\/\RB{}]~\\
    Get the element indexed by {\it index}, failing if {\it index} is out of range.
  \item[\kw{operation} setElement\/\LB{}index \CO \tn{Integer}, value \CO T\/\RB{}]~\\
    Set the element indexed by {\it index} to {\it value}, failing if {\it index} is out of range.
  \item[\kw{function}  upperbound \returns{} \/\LB{}\tn{Integer}\/\RB{}]~\\
    Return the highest valid index.
  \item[\kw{function}  lowerbound \returns{} \/\LB{}\tn{Integer}\/\RB{}]~\\
    Return the lowest valid index.
  \item[\kw{function}  getElement\/\LB{}lb \CO{} \tn{Integer}, length \CO{} \tn{Integer}\/\RB{} \returns{} \/\LB{}aNewArrayType\/\RB{}]~\\
    Return a new \tn{Array}, a, with lower bound {\it lb}, and length {\it length}, such that for lb $\leq$ i $\leq$ lb + length - 1\CO{} self\LB{}i\/\RB{} $==$ a\LB{}i\/\RB{}.
    Fail if lb or lb + length - 1 is out of range.
  \item[\kw{function}  getSlice\/\LB{}lb \CO \tn{Integer}, length \CO \tn{Integer}\/\RB{} \returns{} \/\LB{}aNewArrayType\/\RB{}]~\\
    Same as getElement.
  \item[\kw{operation} setElement\/\LB{}lb \CO{} \tn{Integer}, length \CO{}
  \tn{Integer}, a \CO{} RIS\/\RB{}]~\\
    Set the elements indexed starting at {\it lb} for {\it length} elements,
    so that for each such i in that range\CO{}
    self\/\LB{}i\/\RB{} $==$ a\/\LB{}i\/\RB{}. Fail if
    lb or lb + length - 1 is out of range.
  \item[\kw{operation} setSlice\/\LB{}lb \CO{} \tn{Integer}, length \CO{}
  \tn{Integer}, a \CO{} Sequence.of\/\LB{}T\/\RB{}\/\RB{}]~\\
    Same as setElement.
  \item[\kw{operation} slideTo\/\LB{}newlb \CO{} \tn{Integer}\/\RB{}]~\\
    Change the valid indices for self so that the new lowerbound is {\it newlb}.
  \item[\kw{operation} addUpper\/\LB{}value \CO{} T\/\RB{}]~\\
    Extend the set of valid indices, changing ub to
    ub $+$ 1, and setting the element indexed by the new
    ub to be {\it value}.
  \item[\kw{operation} removeUpper \returns{} \/\LB{}T\/\RB{}]~\\
    Return the element indexed by ub, after contracting
    the set of valid indices to lb $\leq$ i $\leq$ ub $-$ 1.
  \item[\kw{operation} addLower\/\LB{}value \CO{} T\/\RB{}]~\\
    Extend the set of valid indices, changing lb to
    lb $-$ 1, and setting
    the element indexed by the new lb to be {\it value}.
  \item[\kw{operation} removeLower \returns{} \/\LB{}T\/\RB{}]~\\
    Return the element indexed by lb, after contracting
    the set of valid indices to lb $+$ 1 $\leq$ i $\leq$ ub.
  \item[\kw{function}  empty \returns{} \/\LB{}\tn{Boolean}\/\RB{}]~\\
    Return \kw{true} if lb $==$ ub $+$ 1.
  \item[\kw{operation} catenate\/\LB{}a \CO{} RIS\/\RB{} \returns{} \/\LB{}r \CO{} aNewArrayType\/\RB{}]~\\
    Create a new array like self, and then add (using addUpper) each element
    in {\it a} to that new array.
\end{desc}

\subsection{BitChunk}
\label{builtin BitChunk}
BitChunks allow the manipulation of arbitrarily sized sequences of bits.
There are operations to set or retrieve collections of bits at arbitrary bit
positions with lengths up to 32 bits.
BitChunk is a type with the following interface:

\begin{desc}
  \item[\kw{export} \kw{operation} create\/\LB{}n \CO{} \tn{Integer}\/\RB{} \returns{} \/\LB{}Bitchunk\/\RB{}]~\\
    Create a bitChunk large enough to hold {\it n\/} bytes of information.
\end{desc}

An object whose type is BitChunk has the following interface:

\begin{desc}
  \item[\kw{function}  getSigned\/\LB{}off\CO{} \tn{Integer}, len\CO{} \tn{Integer}\/\RB{} \returns{} \/\LB{}\tn{Integer}\/\RB{}]~\\
    Return the bits at offset {\it off} for length {\it len} as a signed
    \tn{Integer} (treat the highest order bit as a sign bit).
  \item[\kw{function}  getUnsigned\/\LB{}off \CO{} \tn{Integer}, len \CO{} \tn{Integer}\/\RB{} \returns{} \/\LB{}\tn{Integer}\/\RB{}]~\\
    Return the bits at offset {\it off} for length {\it len} as an unsigned \tn{Integer}
  \item[\kw{function}  getElement\/\LB{}\tn{Integer}, \tn{Integer}\/\RB{} \returns{} \/\LB{}\tn{Integer}\/\RB{}]~\\
    Equivalent to getUnsigned.
  \item[\kw{operation} setSigned\/\LB{}off\CO{}\tn{Integer}, len\CO{}\tn{Integer}, val\CO{}\tn{Integer}\/\RB{}]~\\
    Set the bits at offset {\it off} for length {\it len} to the low order bits of {\it val}.
  \item[\kw{operation} setUnsigned\/\LB{}\tn{Integer}, \tn{Integer}, \tn{Integer}\/\RB{}]~\\
    Equivalent to setSigned.
  \item[\kw{operation} setElement\/\LB{}\tn{Integer}, \tn{Integer}, \tn{Integer}\/\RB{}]~\\
    Equivalent to setSigned.
  \item[\kw{operation} ntoh\/\LB{}off \CO{} \tn{Integer}, len \CO{} \tn{Integer}\/\RB{}]~\\
    Convert the bits at offset {\it off} with length {\it len} from network to host
    byte order.  Len must be either 16 or 32.
\end{desc}

\subsection{Boolean}
\label{builtin Boolean}
In addition to the operations on Booleans listed here, Booleans are
involved in the evaluation of the conditional and (\kw{and}) and
conditional or (\kw{or}) expressions.
The immutable object Boolean has the following interface:

\begin{desc}
  \item[\kw{function} create\/\LB{}ord \CO{} \tn{Integer}\/\RB{} \returns{} \/\LB{}\tn{Boolean}\/\RB{}]~\\
    If ord is 0 then return {\bf false}, otherwise return {\bf true}.
\end{desc}

\noindent
Objects whose type is \tn{Boolean} are immutable with the following interface:
Comparisons are based on the ordinal values as given by ord

\begin{desc}
  \item[\kw{function} \opd{$>$} \/\LB{}\tn{Boolean}\/\RB{} \returns{} \/\LB{}\tn{Boolean}\/\RB{}]
  \item[\kw{function} \opd{$>=$} \/\LB{}\tn{Boolean}\/\RB{} \returns{} \/\LB{}\tn{Boolean}\/\RB{}]
  \item[\kw{function} \opd{$<$} \/\LB{}\tn{Boolean}\/\RB{} \returns{} \/\LB{}\tn{Boolean}\/\RB{}]
  \item[\kw{function} \opd{$<=$} \/\LB{}\tn{Boolean}\/\RB{} \returns{} \/\LB{}\tn{Boolean}\/\RB{}]
  \item[\kw{function} \opd{$=$} \/\LB{}\tn{Boolean}\/\RB{} \returns{} \/\LB{}\tn{Boolean}\/\RB{}]
  \item[\kw{function} \opd{$!=$} \/\LB{}\tn{Boolean}\/\RB{} \returns{} \/\LB{}\tn{Boolean}\/\RB{}]~\\
  Comparison functions.
  \item[\kw{function} \opd{$\&$} \/\LB{}\tn{Boolean}\/\RB{} \returns{} \/\LB{}\tn{Boolean}\/\RB{}]~\\
    Logical and.
  \item[\kw{function} \opd{$|$} \/\LB{}\tn{Boolean}\/\RB{} \returns{} \/\LB{}\tn{Boolean}\/\RB{}]~\\
    Logical or.
  \item[\kw{function} \opd{$!$}  \returns{} \/\LB{}\tn{Boolean}\/\RB{}]~\\
    Logical negation.
  \item[\kw{function} ord \returns{} \/\LB{}\tn{Integer}\/\RB{}]~\\
    Return 0 when invoked on \kw{false}, 1 when invoked on \kw{true}.
  \item[\kw{function} asString \returns{} \/\LB{}\tn{String}\/\RB{}]~\\
    Return either ``true" or ``false''.
  \item[\kw{function} hash \returns{} \/\LB{}\tn{Integer}\/\RB{}]~\\
    return self.ord
\end{desc}



\subsection{Character}
\label{builtin Character}
\noindent The immutable object Character has the following interface:

\begin{desc}
  \item[\kw{function} Literal\/\LB{}o \CO{} \tn{Integer}\/\RB{} \returns{} \/\LB{}\tn{Character}\/\RB{}]~\\
    Return the \tn{Character} whose ordinal is given by o.
\end{desc}

\noindent Objects whose type is Character are immutable and have the
following interface:

\begin{desc}
  \item[\kw{function} \opd{$>$} \/\LB{}\tn{Character}\/\RB{} \returns{} \/\LB{}\tn{Boolean}\/\RB{}]
  \item[\kw{function} \opd{$>=$} \/\LB{}\tn{Character}\/\RB{} \returns{} \/\LB{}\tn{Boolean}\/\RB{}]
  \item[\kw{function} \opd{$<$} \/\LB{}\tn{Character}\/\RB{} \returns{} \/\LB{}\tn{Boolean}\/\RB{}]
  \item[\kw{function} \opd{$<=$} \/\LB{}\tn{Character}\/\RB{} \returns{} \/\LB{}\tn{Boolean}\/\RB{}]
  \item[\kw{function} \opd{$=$} \/\LB{}\tn{Character}\/\RB{} \returns{} \/\LB{}\tn{Boolean}\/\RB{}]
  \item[\kw{function} \opd{$!=$} \/\LB{}\tn{Character}\/\RB{} \returns{} \/\LB{}\tn{Boolean}\/\RB{}]~\\
  Comparison functions.
  \item[\kw{function} asString \returns{} \/\LB{}\tn{String}\/\RB{}]~\\
    Return a single character string ``c'' when invoked on a character `c'.
  \item[\kw{function} ord \returns{} \/\LB{}\tn{Integer}\/\RB{}]~\\
    Return the character's ordinal number.
  \item[\kw{function} hash \returns{} \/\LB{}\tn{Integer}\/\RB{}]~\\
    Return self.ord.
  \item[\kw{function} isalpha \returns{} \/\LB{}r \CO{} \tn{Boolean}\/\RB{}]~\\
    Return \kw{true} if the character appears in the alphabet.
  \item[\kw{function} isupper \returns{} \/\LB{}r \CO{} \tn{Boolean}\/\RB{}]~\\
    Return \kw{true} if the character is upper case.
  \item[\kw{function} islower \returns{} \/\LB{}r \CO{} \tn{Boolean}\/\RB{}]~\\
    Return \kw{true} if the character is lower case.
  \item[\kw{function} isdigit \returns{} \/\LB{}r \CO{} \tn{Boolean}\/\RB{}]~\\
    Return \kw{true} if the character is a decimal digit.
  \item[\kw{function} isxdigit \returns{} \/\LB{}r \CO{} \tn{Boolean}\/\RB{}]~\\
    Return \kw{true} if the character is a hexadecimal digit.
  \item[\kw{function} isalnum \returns{} \/\LB{}r \CO{} \tn{Boolean}\/\RB{}]~\\
    Return \kw{true} if the character is alphanumeric.
  \item[\kw{function} isspace \returns{} \/\LB{}r \CO{} \tn{Boolean}\/\RB{}]~\\
    Return \kw{true} if the character is white space (blank, tab, newline, etc).
  \item[\kw{function} ispunct \returns{} \/\LB{}r \CO{} \tn{Boolean}\/\RB{}]~\\
    Return \kw{true} if the character is a punctuation mark.
  \item[\kw{function} isprint \returns{} \/\LB{}r \CO{} \tn{Boolean}\/\RB{}]~\\
    Return \kw{true} if the character is printable.
  \item[\kw{function} isgraph \returns{} \/\LB{}r \CO{} \tn{Boolean}\/\RB{}]~\\
    Return \kw{true} if the character is visible when printed.
  \item[\kw{function} iscntrl \returns{} \/\LB{}r \CO{} \tn{Boolean}\/\RB{}]~\\
    Return \kw{true} if the character is a control character.
  \item[\kw{function} toupper \returns{} \/\LB{}r \CO{} \tn{Character}\/\RB{}]~\\
    If the character is a lower case alphabetic, return the upper case
    letter corresponding to it, otherwise return the character itself.
  \item[\kw{function} tolower \returns{} \/\LB{}r \CO{} \tn{Character}\/\RB{}]~\\
    If the character is an upper case alphabetic, return the lower case
    letter corresponding to it, otherwise return the character itself.
\end{desc}

\subsection{ConcreteType}
\label{builtin ConcreteType}
A ConcreteType captures the implementation of some other object.  It
contains all the information required to create and operate on the
collection of objects created from a single object constructor.
ConcreteTypes can only be usefully created by the compiler.
The object ConcreteType is a type and has the following interface:

\begin{desc}
  \item[\kw{operation} create\/\LB{}]~\\
    instanceSize \CO{} \tn{Integer},
    instanceTagMask \CO{} \tn{Integer},
    ops \CO{} COpVector,
    name \CO{} \tn{String},
    filename \CO{} \tn{String},
    template \CO{} \tn{String}\/\RB{} \returns{} \/\LB{}ConcreteType\/\RB{}
\end{desc}

Objects whose type is ConcreteType are immutable and have the following
interface. All of these operations return information describing all of the
objects that could ever be created using this object constructor.

\begin{desc}
  \item[\kw{function}  getInstanceSize \returns{} \/\LB{}\tn{Integer}\/\RB{}]~\\
    Return the size in bytes of the data area of the object.
  \item[\kw{function}  getInstanceTagMask \returns{} \/\LB{}\tn{Integer}\/\RB{}]~\\
    Return a collection of flag bits that include whether the objects are
    immutable and how they are represented by the interpreter.
  \item[\kw{function}  getOps \returns{} \/\LB{}COpVector\/\RB{}]~\\
    Return a list of the operations defined for the objects.
  \item[\kw{function}  getName \returns{} \/\LB{}\tn{String}\/\RB{}]~\\
    Return the name of the object constructor.
  \item[\kw{function}  getFileName \returns{} \/\LB{}\tn{String}\/\RB{}]~\\
    Return the file name of the Emerald source file that contained the
    typeobject from which I was created.
  \item[\kw{function}  getTemplate \returns{} \/\LB{}\tn{String}\/\RB{}]~\\
    Return the template that describes the data area of the
    objects.  The format of the template string is private.
  \item[\kw{function}  getLiterals \returns{} \/\LB{}\tn{ImmutableVectorOfInt}\/\RB{}]~\\
    Return a list of literals that are referenced by the code for operations
    on the object.  The format of this structure is private.
\end{desc}

\label{builtin COpVector}
COpVector is ImmutableVector.of[COpVectorE].

\label{builtin COpVectorE}
A COpVectorE describes an operation.  COpVectorE is a type with
the following interface:

\begin{desc}
  \item[\kw{operation} create\/\LB{}]~\\
    id \CO{} \tn{Integer},
    nArgs \CO{} \tn{Integer},
    nRess \CO{} \tn{Integer},
    name \CO{} \tn{String},
    template \CO{} \tn{String},
    code \CO{} \tn{String}\/\RB{} \returns{} \/\LB{}n \CO{} COpVectorEType\/\RB{}
\end{desc}

Objects whose type is COpVectorE are immutable and have the following
interface:

\begin{desc}
  \item[\kw{function}  getID \returns{} \/\LB{}\tn{Integer}\/\RB{}]~\\
    Return the internal id of the operation.
  \item[\kw{function}  getNArgs \returns{} \/\LB{}\tn{Integer}\/\RB{}]~\\
    Return the number of arguments to the operation.
  \item[\kw{function}  getNRess \returns{} \/\LB{}\tn{Integer}\/\RB{}]~\\
    Return the number of results returned by the operation.
  \item[\kw{function}  getName \returns{} \/\LB{}\tn{String}\/\RB{}]~\\
    Return the name of the operation.
  \item[\kw{function}  getTemplate \returns{} \/\LB{}\tn{String}\/\RB{}]~\\
    Return the template that describes the activation record of the
    operation.  The format of the template string is private.
  \item[\kw{function}  getCode \returns{} \/\LB{}\tn{String}\/\RB{}]~\\
    Return the code for the operation, as a string.
\end{desc}

\label{builtin InterpreterState}
The internal representation of the state of the interpreter.
InterpreterState is a type with the following interface:

\begin{desc}
  \item[\kw{operation} create\/\LB{}]~\\
    pc \CO{} \tn{Integer},
    sp \CO{} \tn{Integer},
    fp \CO{} \tn{Integer},
    sb \CO{} \tn{Integer},
    o \CO{} \tn{Any},
    e \CO{} \tn{Any}\/\RB{}
  \returns{} \/\LB{}InterpreterState\/\RB{}
\end{desc}

Objects whose type is InterpreterState have the following interface:

\begin{desc}
  \item[\kw{function}  getPC \returns{} \/\LB{}\tn{Integer}\/\RB{}]
  \item[\kw{operation} setPC\/\LB{}\tn{Integer}\/\RB{}]
  \item[\kw{function}  getSP \returns{} \/\LB{}\tn{Integer}\/\RB{}]
  \item[\kw{operation} setSP\/\LB{}\tn{Integer}\/\RB{}]
  \item[\kw{function}  getFP \returns{} \/\LB{}\tn{Integer}\/\RB{}]
  \item[\kw{operation} setFP\/\LB{}\tn{Integer}\/\RB{}]
  \item[\kw{function}  getO \returns{} \/\LB{}\tn{Any}\/\RB{}]
  \item[\kw{operation} setO\/\LB{}\tn{Any}\/\RB{}]
  \item[\kw{function}  getSB \returns{} \/\LB{}\tn{Integer}\/\RB{}]
  \item[\kw{operation} setSB\/\LB{}\tn{Integer}\/\RB{}]
  \item[\kw{function}  getE \returns{} \/\LB{}\tn{Any}\/\RB{}]
  \item[\kw{operation} setE\/\LB{}\tn{Any}\/\RB{}]
\end{desc}

PC is the program counter, SP is the stack pointer, FP is the frame pointer,
and SB is the base of the stack.  O is the current object, and E is the
environment of the current process.  Each Emerald process has a per-
process environment which can be used at the programmer's discretion.

\subsection{Condition}
\label{builtin Condition}

A condition object may only be used within the monitor within
which it was created.
The object Condition is immutable, and has the following interface:

\begin{desc}
  \item[\kw{operation} create \returns{} \/\LB{}\tn{Condition}\/\RB{}]
\end{desc}

Objects whose type is Condition have no operations;  wait, signal, and
awaiting are language primitives.

\subsection{InStream}
\label{builtin instream}
InStream objects provide the ability to read files.  The InStream object is
immutable and has the following interface:

\begin{desc}
  \item[\kw{operation} fromUnix\/\LB{}fn \CO{} \tn{String}, mode \CO{} \tn{String}\/\RB{} \returns{} \/\LB{}InStream\/\RB{}]~\\
    Return a new input stream attached to the given operating system file.
    Mode is as in fopen in C, and must begin with `r'.
    The operation fails if the file does not exist or cannot be opened.
  \item[\kw{operation} create\/\LB{}file \CO{} \tn{Integer}\/\RB{} \returns{} \/\LB{}InStream\/\RB{}]~\\
    Return a new input stream attached to the given file descriptor.
\end{desc}

\noindent
All of the input operations on InStream objects fail if the operation cannot
be performed.  In particular, they fail upon reaching end-of-file or if the
stream has been closed.
Objects whose type is InStream have the following interface:

\begin{desc}
  \item[\kw{operation} getChar \returns{} \/\LB{}\tn{Character}\/\RB{}]~\\
    Return the next character from the input source.  Fail if the stream is
    closed or eos has been reached.
  \item[\kw{operation} unGetChar\/\LB{}c \CO{} \tn{Character}\/\RB{}]~\\
    Push the character on the the front of the input stream.  There is no
    requirement that the character be one previously read from the stream.
    Only the amount of available memory limits the amount of data that can
    be pushed back on a stream.
  \item[\kw{operation} getString \returns{} \/\LB{}\tn{String}\/\RB{}]~\\
    Return one line of input, including the terminating newline
    character, if any.
  \item[\kw{function} eos \returns{} \/\LB{}\tn{Boolean}\/\RB{}]~\\
    Return \kw{true} if the end of the input stream has been reached.  Note
    that this predicate will block for terminal input on a terminal stream
    if necessary in order to determine whether the end of the stream has
    been reached.
  \item[\kw{operation} close]~\\
    Close the stream.
  \item[\kw{function} isAtty \returns{} \/\LB{}\tn{Boolean}\/\RB{}]~\\
    Return \kw{true} if the underlying Unix file is a tty.
  \item[\kw{operation} fillVector\/\LB{}VectorOfChar\/\RB{} \returns{} \/\LB{}Integer\/\RB{}]~\\
    Reads up to one line from the input, placing the characters in the
    provided vector.  Returns the number of characters read.  The vector
    will contain the newline character that caused reading to terminate
    unless no newline
    character is read before the vector is full.
  \item[\kw{operation} rawRead\/\LB{}VectorOfChar\/\RB{} \returns{} \/\LB{}Integer\/\RB{}]~\\
    Reads from the input placing the characters in the
    provided vector.  Returns the number of characters read.
\end{desc}

\subsection{Integer}
\label{builtin Integer}
Conversion between integers and reals is accomplished by the asReal
operation on Integers, and the asInteger operation on Reals.
The immutable object Integer has the following interface:

\begin{desc}
  \item[\kw{function} literal\/\LB{}rep \CO{} \tn{String}\/\RB{} \returns{} \/\LB{}\tn{Integer}\/\RB{}]~\\
    Return the integer parsed from the front of the string {\it rep}.  The C
    function strtol is used to parse the integer.
\end{desc}

\noindent
Objects whose type is Integer are immutable, and have the following
interface.  For those operations that manipulate bits, bit numbering starts at
0, which represents the high order bit of the integer.

\begin{desc}
  \item[\kw{function} \opd{$+$} \/\LB{}\tn{Integer}\/\RB{} \returns{} \/\LB{}\tn{Integer}\/\RB{}]
  \item[\kw{function} \opd{$-$} \/\LB{}\tn{Integer}\/\RB{} \returns{} \/\LB{}\tn{Integer}\/\RB{}]
  \item[\kw{function} \opd{$*$} \/\LB{}\tn{Integer}\/\RB{} \returns{} \/\LB{}\tn{Integer}\/\RB{}]
  \item[\kw{function} \opd{$/$} \/\LB{}\tn{Integer}\/\RB{} \returns{} \/\LB{}\tn{Integer}\/\RB{}]
  \item[\kw{function} \opd{\#} \/\LB{}\tn{Integer}\/\RB{} \returns{} \/\LB{}\tn{Integer}\/\RB{}]~\\
    Arithmetic functions.  \# represents modulus.
  \item[\kw{function} \opd{$>$} \/\LB{}\tn{Integer}\/\RB{} \returns{} \/\LB{}\tn{Boolean}\/\RB{}]
  \item[\kw{function} \opd{$>=$} \/\LB{}\tn{Integer}\/\RB{} \returns{} \/\LB{}\tn{Boolean}\/\RB{}]
  \item[\kw{function} \opd{$<$} \/\LB{}\tn{Integer}\/\RB{} \returns{} \/\LB{}\tn{Boolean}\/\RB{}]
  \item[\kw{function} \opd{$<=$} \/\LB{}\tn{Integer}\/\RB{} \returns{} \/\LB{}\tn{Boolean}\/\RB{}]
  \item[\kw{function} \opd{$=$} \/\LB{}\tn{Integer}\/\RB{} \returns{} \/\LB{}\tn{Boolean}\/\RB{}]
  \item[\kw{function} \opd{$!=$} \/\LB{}\tn{Integer}\/\RB{} \returns{} \/\LB{}\tn{Boolean}\/\RB{}]~\\
  Comparison functions.
  \item[\kw{function} \opd{\tt\mytilde} \returns{} \/\LB{}\tn{Integer}\/\RB{}]~\\
    Negation.
  \item[\kw{function} \opd{$-$} \returns{} \/\LB{}\tn{Integer}\/\RB{}]~\\
    Negation, identical to {\tt\mytilde}.
  \item[\kw{function} asString \returns{} \/\LB{}\tn{String}\/\RB{}]~\\
    Return a \tn{String} representing the value of the \tn{Integer} with no
    leading zeros in decimal.
  \item[\kw{function} hash \returns{} \/\LB{}\tn{Integer}\/\RB{}]~\\
    Return self.
  \item[\kw{function} abs \returns{} \/\LB{}\tn{Integer}\/\RB{}]~\\
    Return the absolute value of self.
  \item[\kw{function} asReal \returns{} \/\LB{}\tn{Real}\/\RB{}]~\\
    Return a \tn{Real} value representing the same value as self.
  \item[\kw{function} \opd{$\&$} \/\LB{}other \CO{} \tn{Integer}\/\RB{} \returns{} \/\LB{}\tn{Integer}\/\RB{}]~\\
    Return the bitwise and of self and {\it other}.
  \item[\kw{function} \opd{$|$} \/\LB{}other \CO{} \tn{Integer}\/\RB{} \returns{} \/\LB{}\tn{Integer}\/\RB{}]~\\
    Return the bitwise or of self and {\it other}.
  \item[\kw{function} setBit\/\LB{}o \CO{} \tn{Integer}, v \CO{} \tn{Boolean}\/\RB{} \returns{} \/\LB{}r \CO{} \tn{Integer}\/\RB{}]~\\
    Return an \tn{Integer} like self except in the {\it o}'th bit position,
    where it has the bit {\it v}.
  \item[\kw{function} getBit\/\LB{}o \CO{} \tn{Integer}\/\RB{} \returns{} \/\LB{}r \CO{} \tn{Boolean}\/\RB{}]~\\
    Return the bit in {\it o}'th bit position.
  \item[\kw{function} setBits\/\LB{}o \CO{} \tn{Integer}, l \CO{} \tn{Integer}, v \CO{} \tn{Integer}\/\RB{} \returns{} \/\LB{}r \CO{} \tn{Integer}\/\RB{}]~\\
    Return an \tn{Integer} like self except in the {\it o} through $o+l-1$'th positions, where
    the bits represent the low order bits of the \tn{Integer} v.
  \item[\kw{function} getBits\/\LB{}o \CO{} \tn{Integer}, l \CO{} \tn{Integer}\/\RB{} \returns{} \/\LB{}r \CO{} \tn{Integer}\/\RB{}]~\\
    Return the bits in the {\it o} through $o+l-1$'th bit positions, without
    sign extension.
\end{desc}

\subsection{Node, NodeList, NodeListElement, Directory, Handler}
\label{builtin Node}
The nodeEventHandler entries allow appropriate operations to be
invoked when the node detects changes in the network topology.  The
operations that query network topology use the auxiliary types
NodeList and NodeListElement which are described below.
The object Node is a type with the following interface:

\begin{desc}
  \item[\kw{operation} getStdin \returns{} \/\LB{}InStream\/\RB{}]~\\
    Return an InStream representing the current \tn{Node}'s standard input.
  \item[\kw{operation} getStdout \returns{} \/\LB{}OutStream\/\RB{}]~\\
    Return an OutStream representing the current \tn{Node}'s standard output.
\end{desc}

\noindent Objects with type Node are mutable, are fixed at their initial
locations, and have the following interface:

\begin{desc}
  \item[\kw{operation} getActiveNodes \returns{} \/\LB{}NodeList\/\RB{}]~\\
    Return a list containing information about all nodes in the Emerald
    environment that are known to be functioning.
  \item[\kw{operation} getAllNodes \returns{} \/\LB{}NodeList\/\RB{}]~\\
    Return a list containing information about all nodes in the Emerald
    environment, whether they are functioning or not.
  \item[\kw{operation} getNodeInformation \returns{} \/\LB{}NodeListElement\/\RB{}]~\\
    Return information about the node.  This is not currently implemented.
  \item[\kw{operation} getTimeOfDay \returns{} \/\LB{}Time\/\RB{}]~\\
    Return an object of type \tn{Time } representing the current wall clock
    time.
  \item[\kw{operation} delay\/\LB{}howlong \CO{} Time\/\RB{}]~\\
    Put the current process to sleep until the amount of time specified by
    {\it howlong} has passed.
  \item[\kw{operation} waitUntil\/\LB{}untilwhen \CO{} Time\/\RB{}]~\\
    Put the current process to sleep until the absolute time specified by
    {\it untilwhen} has arrived.
  \item[\kw{operation} getLoadAverage \returns{} \/\LB{}\tn{Real}\/\RB{}]~\\
    Return a \tn{Real} number representing the load on the machine on which
    the target node is executing.  This is not currently implemented.
  \item[\kw{operation} setNodeEventHandler\/\LB{}h \CO{} HandlerType\/\RB{}]~\\
    Register {\it h} as an additional handler to receive notification when
    nodes come up or do down.
  \item[\kw{operation} removeNodeEventHandler\/\LB{}h \CO{} HandlerType\/\RB{}]~\\
    Unregister h as a node event handler.
  \item[\kw{operation} setNearbyNodeEventHandler\/\LB{}h \CO{} HandlerType\/\RB{}]~\\
    Register {\it h} as an additional handler to receive notification when
    nearby nodes are discovered.
  \item[\kw{operation} removeNearbyNodeEventHandler\/\LB{}h \CO{} HandlerType\/\RB{}]~\\
    Unregister h as a nearby node event handler. 
  \item[\kw{operation} getStdin \returns{} \/\LB{}InStream\/\RB{}]~\\
    Return an InStream representing the target \tn{Node}'s standard input.
  \item[\kw{operation} getStdout \returns{} \/\LB{}OutStream\/\RB{}]~\\
    Return an OutStream representing the target \tn{Node}'s standard output.
  \item[\kw{function} getLNN \returns{} \/\LB{}\tn{Integer}\/\RB{}]~\\
    Return the current node's Logical Node Number.  This value is not
    interesting.
  \item[\kw{function} getName \returns{} \/\LB{}\tn{String}\/\RB{}]~\\
    Return a string containing the name of the current host.  If the current
    host name cannot be determined (as on DOS) return a string containing as
    much information as is available.
  \item[\kw{function} getRootDirectory \returns{} \/\LB{}\tn{Directory}\/\RB{}]~\\
    Return the root directory of the Emerald universe.  Each cooperationg
    collection of Emerald nodes has a single root directory which is the
    root of the name service.
\end{desc}

\noindent A NodeListElement is an immutable record with 4 fields that
provide information about a node:

\begin{desc}
  \item[\kw{function} getTheNode \returns{} \/\LB{}\tn{Node}\/\RB{}]
  \item[\kw{function} getUp \returns{} \/\LB{}\tn{Boolean}\/\RB{}]
  \item[\kw{function} getIncarnationTime \returns{} \/\LB{}Time\/\RB{}]
  \item[\kw{function} getLNN \returns{} \/\LB{}\tn{Integer}\/\RB{}]
\end{desc}

\noindent NodeList is
{\it ImmutableVector.of\/\LB{}NodeListElement\/\RB{}}.

Directory is the type of name service directories.  Directory is a type with
interface:
\begin{desc}
  \item[\kw{operation} create \returns{} \/\LB{}r \CO{} Directory\/\RB{}]~\\
    Return a new directory.
\end{desc}

\noindent Objects whose type is Directory have the following
interface:

\begin{desc}
  \item[\kw{operation} insert\/\LB{}name \CO{} \tn{String}, value \CO{} \tn{Any}\/\RB{}]~\\
    Insert the object {\it value} in the directory under the name {\it name}.
  \item[\kw{function} lookup\/\LB{}name \CO{} \tn{String}\/\RB{} \returns{} \/\LB{}\tn{Any}\/\RB{}]~\\
    Return the object in the directory stored under the name {\it name}.  If
    there is no object stored under that name, return \kw{nil}.
  \item[\kw{operation} delete\/\LB{}\tn{String}\/\RB{}]~\\
    Delete the object in the directory stored under the name {\it name}.
    It is not an error to delete a nonexistent entry.
  \item[\kw{function} list \returns{} \/\LB{}ImmutableVectorOfString\/\RB{}]~\\
    Return a list of all of the names defined in the directory.
\end{desc}

Handler represents the type of objects that can register to be informed when
Nodes come up and go down.  Such objects must conform to the type Handler,
which is:

{\it\begin{minipage}{\textwidth}\begin{tabbing}
xxx\=xxx\=xxx\=xxx\=xxx\=xxx\=xxx\=xxx\=xxx\=xxx\=xxx\=xxx\=xxx\=\+\kill%
\kw{typeobject} HandlerType\+\\*{}%
  \kw{operation} nodeUp\/\LB{}\tn{Node}, Time\/\RB{}\\*{}%
  \kw{operation} nodeDown\/\LB{}\tn{Node}, Time\/\RB{}\-\\*{}%
\kw{end} HandlerType
\end{tabbing}\end{minipage}}

\subsection{None}
\label{builtin None}
None is the type that supports all operations, and is therefore
implemented only by the \kw{nil} object.  It is defined to complete the
lattice structure of \emd{} types; None represents the top element of the
type lattice.

\subsection{OutStream}
\label{builtin outstream}
OutStream objects provide the ability to write files.  The OutStream object
is a type with the following interface:

\begin{desc}
  \item[\kw{operation} toUnix\/\LB{}fn \CO{} \tn{String}, mode \CO{} \tn{String}\/\RB{} \returns{} \/\LB{}OutStream\/\RB{}]~\\
    Return a new output stream attached to the given operating system file.
    Mode is as in fopen in C, and must begin with `w' or `a'.
    The operation fails if the file cannot be opened in the given mode.
  \item[\kw{operation} create\/\LB{}file \CO{} \tn{Integer}\/\RB{} \returns{} \/\LB{}OutStream\/\RB{}]~\\
    Return a new output stream attached to the given file descriptor.
\end{desc}

\noindent Objects with type OutStream have the following interface.
All of the output operations on OutStream objects fail if the requested
operation cannot be performed.

\begin{desc}
  \item[\kw{operation} putChar\/\LB{}c \CO{} \tn{Character}\/\RB{}]~\\
    Append the character {\it c} on the stream.
  \item[\kw{operation} putInt\/\LB{}n \CO{} \tn{Integer}, width \CO{} \tn{Integer}\/\RB{}]~\\
    Append the decimal string representation of the \tn{Integer} {\it n} on the
    stream, right justified in a field {\it width} wide.  If the value n
    cannot be accurately represented in width characters, then the width
    specification will be treated as infinity.
  \item[\kw{operation} writeInt\/\LB{}n \CO{} \tn{Integer}, size \CO{} \tn{Integer}\/\RB{}]~\\
    Write the binary representation of the \tn{Integer} {\it n} in {\it
    size} bytes, in network byte order.  Size must be 1, 2, or 4.
  \item[\kw{operation} putReal\/\LB{}x \CO{} \tn{Real}\/\RB{}]~\\
    Append the decimal string representation of the \tn{Real} {\it x} on the
    stream.  A representation of the value of x will be chosen so as to not
    lose accuracy.
  \item[\kw{operation} putString\/\LB{}s \CO{} \tn{String}\/\RB{}]~\\
    Append the \tn{String} {\it s} on the stream.
  \item[\kw{operation} flush]~\\
    Flush any buffered output.
  \item[\kw{operation} close]~\\
    Close the stream, flushing any buffered output and preventing any
    further output.
\end{desc}

\subsection{Real}
\label{builtin Real}
The Real type is implemented as a 32-bit floating-point number.
The object Real is a type with the following interface:

\begin{desc}
  \item[\kw{function} literal\/\LB{}s \CO{} \tn{String}\/\RB{} \returns{} \/\LB{}\tn{Real}\/\RB{}]~\\
    Return a Real number parsed from the string {\it s}.  The C language
    function atof is used to parse the string.
\end{desc}

\noindent Objects with type Real are immutable and have the following
interface:

\begin{desc}
  \item[\kw{function} \opd{$+$} \/\LB{}\tn{Real}\/\RB{} \returns{} \/\LB{}\tn{Real}\/\RB{}]
  \item[\kw{function} \opd{$-$} \/\LB{}\tn{Real}\/\RB{} \returns{} \/\LB{}\tn{Real}\/\RB{}]
  \item[\kw{function} \opd{$*$} \/\LB{}\tn{Real}\/\RB{} \returns{} \/\LB{}\tn{Real}\/\RB{}]
  \item[\kw{function} \opd{$/$} \/\LB{}\tn{Real}\/\RB{} \returns{} \/\LB{}\tn{Real}\/\RB{}]~\\
  Arithmetic functions.
  \item[\kw{function} \opd{$\myuparrow$} \/\LB{}\tn{Real}\/\RB{} \returns{} \/\LB{}\tn{Real}\/\RB{}]~\\
  Exponentiation. $a \myuparrow{} b$ returns $a$ raised to the exponent $b$.
  \item[\kw{function} \opd{$>$} \/\LB{}\tn{Real}\/\RB{} \returns{} \/\LB{}\tn{Boolean}\/\RB{}]
  \item[\kw{function} \opd{$>=$} \/\LB{}\tn{Real}\/\RB{} \returns{} \/\LB{}\tn{Boolean}\/\RB{}]
  \item[\kw{function} \opd{$<$} \/\LB{}\tn{Real}\/\RB{} \returns{} \/\LB{}\tn{Boolean}\/\RB{}]
  \item[\kw{function} \opd{$<=$} \/\LB{}\tn{Real}\/\RB{} \returns{} \/\LB{}\tn{Boolean}\/\RB{}]
  \item[\kw{function} \opd{$=$} \/\LB{}\tn{Real}\/\RB{} \returns{} \/\LB{}\tn{Boolean}\/\RB{}]
  \item[\kw{function} \opd{$!=$} \/\LB{}\tn{Real}\/\RB{} \returns{} \/\LB{}\tn{Boolean}\/\RB{}]~\\
  Comparison functions.
  \item[\kw{function} \opd{\tt\mytilde} \returns{} \/\LB{}\tn{Real}\/\RB{}]~\\
    Return the negation of self.
  \item[\kw{function} \opd{$-$} \returns{} \/\LB{}\tn{Real}\/\RB{}]~\\
    Return the negation of self, same as {\tt\mytilde}.
  \item[\kw{function} asString \returns{} \/\LB{}\tn{String}\/\RB{}]~\\
    Return a string representing my value.  The C language function
    sprintf's \cd{}g specification is used.
  \item[\kw{function} asInteger \returns{} \/\LB{}\tn{Integer}\/\RB{}]~\\
    Return an integer as close as possible to, but lower than my value.
\end{desc}

\subsection{Sequence, SequenceOfAny, SequenceOfString}
Sequence represents indexable sequences.  The object Sequence is immutable
and has interface:
\begin{desc}
  \item[\kw{function} of\/\LB{}T \CO{} \tn{type}\/\RB{} \returns{}
  \/\LB{}aNewSequenceType \CO{} Type\/\RB{} \kw{forall} T]
\end{desc}

\noindent The object resulting from {\it \tn{Sequence}.of\/\LB{}T\/\RB{}}
is a \tn{Signature} defined as follows:

{\it\begin{minipage}{\textwidth}\begin{tabbing}
xxx\=xxx\=xxx\=xxx\=xxx\=xxx\=xxx\=xxx\=xxx\=xxx\=xxx\=xxx\=xxx\=\+\kill%
\kw{typeobject} aNewSequence\+\\*{}%
  \kw{function} lowerbound \returns{} \/\LB{}\tn{Integer}\/\RB{}\\*{}%
  \kw{function} upperbound \returns{} \/\LB{}\tn{Integer}\/\RB{}\\*{}%
  \kw{function} getElement\/\LB{}\tn{Integer}\/\RB{} \returns{} \/\LB{}T\/\RB{}\-\\*{}%
\kw{end} aNewSequence
\end{tabbing}\end{minipage}}

\label{builtin SequenceOfAny}
SequenceOfAny is Sequence of Any.
Vector.of[Any], ImmutableVector.of[Any], and Array.of[Any]
all conform to SequenceOfAny.

\label{builtin SequenceOfCharacter}
SequenceOfCharacter is Sequence of Character.
Vector.of[Character], ImmutableVector.of[Character], and Array.of[Character]
all conform to SequenceOfCharacter.

\subsection{Signature, AOpVector, AOpVectorE, AParamList}
\label{builtin Signature}
Signature is the type of the object constructed by the compiler as the
result of a typeobject constructor.  Signature has no interesting operations.

\noindent Objects whose type is Signature are types, and have the
following interface, all of which operations return information about the
typeobject constructor that was used in the creation of the signature.

\begin{desc}
  \item[\kw{function} getIsImmutable \returns{} \/\LB{}\tn{Boolean}\/\RB{}]~\\
    Return \kw{true} if the typeobject is immutable.
  \item[\kw{function} getIsTypeVariable \returns{} \/\LB{}\tn{Boolean}\/\RB{}]~\\
    Return \kw{true} if the typeobject is a type variable which resulted
    from a for all or such that clause.
  \item[\kw{function} getOps \returns{} \/\LB{}AOpVector\/\RB{}]~\\
    Return a list of the operations in the typeobject.
  \item[\kw{function} getName \returns{} \/\LB{}\tn{String}\/\RB{}]~\\
    Return the name of the typeobject.
  \item[\kw{function} getFileName \returns{} \/\LB{}\tn{String}\/\RB{}]~\\
    Return the file name of the Emerald source file that contained the
    typeobject from which I was created.
\end{desc}

\label{builtin AParamList}
AParamList is ImmutableVector.of[Signature].

\label{builtin AOpVector}
AOpVector is ImmutableVector.of[AOpVectorE].

\label{builtin AOpVectorE}
An AOpVectorE describes an operation in a Signature.  AOpVectorE is a type with
the following interface:

\begin{desc}
  \item[\kw{operation} create\/\LB{}]~\\
    id \CO{} \tn{Integer},
    NArgs \CO{} \tn{Integer},
    NRess \CO{} \tn{Integer},
    isFunction \CO{} \tn{Boolean},
    name \CO{} \tn{String},
    arguments \CO{} AParamList,
    results \CO{} AParamList\/\RB{} \returns{} \/\LB{}AOpVectorE\/\RB{}
\end{desc}

Objects whose type is AOpVectorE are immutable and have the following
interface:

\begin{desc}
  \item[\kw{function}  getID \returns{} \/\LB{}\tn{Integer}\/\RB{}]~\\
    Return the internal id of the operation.
  \item[\kw{function}  getNArgs \returns{} \/\LB{}\tn{Integer}\/\RB{}]~\\
    Return the number of arguments to the operation.
  \item[\kw{function}  getNRess \returns{} \/\LB{}\tn{Integer}\/\RB{}]~\\
    Return the number of results returned by the operation.
  \item[\kw{function}  getIsFunction \returns{} \/\LB{}\tn{Boolean}\/\RB{}]~\\
    Return \kw{true} if the operation is a function.
  \item[\kw{function}  getName \returns{} \/\LB{}\tn{String}\/\RB{}]~\\
    Return the name of the operation.
  \item[\kw{function}  getArguments \returns{} \/\LB{}AParamList\/\RB{}]~\\
    Return a list of the argument types.
  \item[\kw{function}  getResults   \returns{} \/\LB{}AParamList\/\RB{}]~\\
    Return a list of the result types.
\end{desc}

\subsection{String}
\label{builtin String}

The object String is a type and has the following interface:

\begin{desc}
  \item[\kw{operation} Literal\/\LB{}rep \CO{} Sequence.of\/\LB{}Character\/\RB{}, offset \CO{} \tn{Integer}, len \CO{} \tn{Integer}\/\RB{} \returns{} \/\LB{}\tn{String}\/\RB{}]~\\
    Return a \tn{String} with the characters from the given sequence of
    characters.
  \item[\kw{operation} FLiteral\/\LB{}rep \CO{} VectorOfChar, offset \CO{} \tn{Integer}, len \CO{} \tn{Integer}\/\RB{} \returns{} \/\LB{}\tn{String}\/\RB{}]~\\
    Same as literal, but with a more restrictive argument type (it must be a
    \tn{Vector}), which makes the implementation more efficient.
\end{desc}

\noindent Objects whose type is String are immutable and have the following
interface:

\begin{desc}
  \item[\kw{function} \opd{$>$} \/\LB{}\tn{String}\/\RB{} \returns{} \/\LB{}\tn{Boolean}\/\RB{}]
  \item[\kw{function} \opd{$>=$} \/\LB{}\tn{String}\/\RB{} \returns{} \/\LB{}\tn{Boolean}\/\RB{}]
  \item[\kw{function} \opd{$<$} \/\LB{}\tn{String}\/\RB{} \returns{} \/\LB{}\tn{Boolean}\/\RB{}]
  \item[\kw{function} \opd{$<=$} \/\LB{}\tn{String}\/\RB{} \returns{} \/\LB{}\tn{Boolean}\/\RB{}]
  \item[\kw{function} \opd{$=$} \/\LB{}\tn{String}\/\RB{} \returns{} \/\LB{}\tn{Boolean}\/\RB{}]
  \item[\kw{function} \opd{$!=$} \/\LB{}\tn{String}\/\RB{} \returns{} \/\LB{}\tn{Boolean}\/\RB{}]~\\
    Comparison functions.
  \item[\kw{function} getElement\/\LB{}index \CO{} \tn{Integer}\/\RB{} \returns{} \/\LB{}\tn{Character}\/\RB{}]~\\
    Return the character at position {\it index}, numbering from 0.
  \item[\kw{function} getSlice\/\LB{}lb \CO{} \tn{Integer}, length \CO{} \tn{Integer}\/\RB{} \returns{} \/\LB{}\tn{String}\/\RB{}]~\\
    Return a substring of the string, starting at position {\it lb}, for
    {\it length} characters.
  \item[\kw{function} getElement\/\LB{}lb \CO{} \tn{Integer}, length \CO{} \tn{Integer}\/\RB{} \returns{} \/\LB{}\tn{String}\/\RB{}]~\\
    Same as getSlice, but can use the convenient subscript syntax.
  \item[\kw{function} length \returns{} \/\LB{}\tn{Integer}\/\RB{}]~\\
    Return the length of the string.
  \item[\kw{function} \opd{$|$$|$} \/\LB{}more \CO{} \tn{String}\/\RB{} \returns{} \/\LB{}\tn{String}\/\RB{}]~\\
    Return the result of concatenating the string {\it more} at the end of
    self.
  \item[\kw{function} asString \returns{} \/\LB{}\tn{String}\/\RB{}]~\\
    Return self.
  \item[\kw{function} lowerbound \returns{} \/\LB{}\tn{Integer}\/\RB{}]~\\
    Return 0.
  \item[\kw{function} upperbound \returns{} \/\LB{}\tn{Integer}\/\RB{}]~\\
    Return self.length $-$ 1
  \item[\kw{function} hash \returns{} \/\LB{}\tn{Integer}\/\RB{}]~\\
    Return a characteristic \tn{Integer}.  The hash function is chosen to
    make the probability of two distinct strings having the same hash value
    quite small.
  \item[\kw{function} index\/\LB{}ch \CO{} \tn{Character}\/\RB{} \returns{} \/\LB{}r \CO{} \tn{Integer}\/\RB{}]~\\
    Return the first position in the string at which the character {\it ch}
    appears.  If ch does not appear, then return \kw{nil}.
  \item[\kw{function} rindex\/\LB{}ch \CO{} \tn{Character}\/\RB{} \returns{}
	  \/\LB{}r \CO{} \tn{Integer}\/\RB{}]~\\
    Return the last position in the string at which the character {\it ch}
    appears.  If ch does not appear, then return \kw{nil}.
  \item[\kw{function} span\/\LB{}s \CO{} \tn{String}\/\RB{} \returns{} \/\LB{}r \CO{} \tn{Integer}\/\RB{}]~\\
    Return the first position in the string at which a character not in the
    string {\it s} appears.  If all characters are in s, then return the
    length of the string.
  \item[\kw{function} cspan\/\LB{}s \CO{} \tn{String}\/\RB{} \returns{} \/\LB{}r \CO{} \tn{Integer}\/\RB{}]~\\
    Return the first position in the string at which a character in the
    string {\it s} appears.  If no character in s appears in the string,
    then return the length of the string.
  \item[\kw{function} str\/\LB{}s \CO{} \tn{String}\/\RB{} \returns{} \/\LB{}r \CO{} \tn{Integer}\/\RB{}]~\\
    Return the first position in the string at which the string {\it s}
    appears as a substring.  If s is not a substring of the string, then
    return \kw{nil}.
  \item[\kw{operation} token\/\LB{}sep \CO{} \tn{String}\/\RB{} \returns{} \/\LB{}token \CO{} \tn{String}, rest\CO{} \tn{String}\/\RB{}]~\\
    Parse a token from the front of the string.  Leading characters that
    appear in the string {\it sep} are skipped, the first sequence of
    characters that do not appear in the string sep are returned in the
    result variable {\it token}.  If any additional characters appear in the
    string after the token, then return the rest of the string after the token
    in the result variable {\it rest}, otherwise return \kw{nil} in rest.
    If there is no token in the string (all characters are in the string
    sep), then return \kw{nil} as both token and rest.
\end{desc}


\subsection{Time}
\label{builtin Time}
Times represent times and dates.  They are stored as a number of
seconds (since Jan 1, 1970 when interpreted as dates) and a number of
microseconds.  They can be
used as either dates or times, and the standard arithmetic operations
are defined on them (where they make sense).  The object Time is a type
and has the following interface:

\begin{desc}
  \item[\kw{operation} create\/\LB{}seconds \CO{} \tn{Integer}, microseconds \CO{} \tn{Integer}\/\RB{} \returns{} \/\LB{}Time\/\RB{}]~\\
    Create a new time object with the given values for its seconds and
    microseconds fields.
\end{desc}

\noindent Objects with type Time are immutable and have the following
interface:

\begin{desc}
  \item[\kw{function} \opd{$+$} \/\LB{}Time\/\RB{} \returns{} \/\LB{}Time\/\RB{}]
  \item[\kw{function} \opd{$-$} \/\LB{}Time\/\RB{} \returns{} \/\LB{}Time\/\RB{}]
  \item[\kw{function} \opd{$*$} \/\LB{}n \CO{} \tn{Integer}\/\RB{} \returns{} \/\LB{}Time\/\RB{}]
  \item[\kw{function} \opd{$/$} \/\LB{}\tn{Integer}\/\RB{} \returns{} \/\LB{}Time\/\RB{}]~\\
    Arithmetic functions.
  \item[\kw{function} \opd{$>$} \/\LB{}Time\/\RB{} \returns{} \/\LB{}\tn{Boolean}\/\RB{}]
  \item[\kw{function} \opd{$>=$} \/\LB{}Time\/\RB{} \returns{} \/\LB{}\tn{Boolean}\/\RB{}]
  \item[\kw{function} \opd{$<$} \/\LB{}Time\/\RB{} \returns{} \/\LB{}\tn{Boolean}\/\RB{}]
  \item[\kw{function} \opd{$<=$} \/\LB{}Time\/\RB{} \returns{} \/\LB{}\tn{Boolean}\/\RB{}]
  \item[\kw{function} \opd{$=$} \/\LB{}Time\/\RB{} \returns{} \/\LB{}\tn{Boolean}\/\RB{}]
  \item[\kw{function} \opd{$!=$} \/\LB{}Time\/\RB{} \returns{} \/\LB{}\tn{Boolean}\/\RB{}]~\\
    Comparison functions.
  \item[\kw{function} getSeconds \returns{} \/\LB{}\tn{Integer}\/\RB{}]~\\
    Return the seconds component.
  \item[\kw{function} getMicroSeconds \returns{} \/\LB{}\tn{Integer}\/\RB{}]~\\
    Return the microseconds component.
  \item[\kw{function} asString \returns{} \/\LB{}\tn{String}\/\RB{}]~\\
    Return a string representing the number of seconds and microseconds
    formatted in decimal, as in 6:002342.
  \item[\kw{function} asDate \returns{} \/\LB{}\tn{String}\/\RB{}]~\\
    Return a string representing the value of the time as a date.  The
    string has the form: Sat Aug 17 20:19:50 PDT 1996.
\end{desc}

\subsection{Type}
\label{builtin Type}
The object \tn{Type} represents the type of all types.  It has no other
useful operations.

\subsection{Vector, VectorOfInt, VectorOfChar, }
\label{builtin Vector}
Vector is the name of an object with a polymorphic operation of that creates
mutable vectors.  Vectors provide the most
primitive mechanism for acquiring indexable storage.  The builtin
object Array is implemented entirely in Emerald, using the facilities
offered by Vector.  The object Vector is immutable and has the following
interface:

\begin{desc}
  \item[\kw{function} of\/\LB{}T \CO{} \tn{Type}\/\RB{} \returns{} \/\LB{}aNewVectorCreatorType\/\RB{}]
    \kw{forall} T
\end{desc}

\noindent The creator object resulting from {\it \tn{Vector}.of\/\LB{}T\/\RB{}}
is a type with the following interface, named aNewVectorCreatorType:

\begin{desc}
  \item[\kw{operation} create\/\LB{}length \CO{} \tn{Integer}\/\RB{} \returns{} \/\LB{}aNewVectorType\/\RB{}]~\\
    Create a new vector of length {\it length}, all of whose elements are
    \kw{nil}.
\end{desc}

\noindent Objects with type {\it \tn{Vector}.of\/\LB{}T\/\RB{}} have the
following interface, called aNewVectorType:

\begin{desc}
  \item[\kw{function}  getElement\/\LB{}index \CO{} \tn{Integer}\/\RB{} \returns{} \/\LB{}T\/\RB{}]~\\
    Return the element at {\it index}.
  \item[\kw{operation} setElement\/\LB{}index \CO{} \tn{Integer}, value \CO{} T\/\RB{}]~\\
    Set the element at {\it index} to {\it value}.
  \item[\kw{function}  lowerbound \returns{} \/\LB{}\tn{Integer}\/\RB{}]~\\
    Return 0.
  \item[\kw{function}  upperbound \returns{} \/\LB{}\tn{Integer}\/\RB{}]~\\
    Return the largest valid index.
  \item[\kw{function}  getSlice\/\LB{}lb \CO{} \tn{Integer}, length\CO{} \tn{Integer}\/\RB{} \returns{} \/\LB{}aNewVectorType\/\RB{}]~\\
    Return a new object with type aNewVectorType containing the {\it length}
    elements starting at position {\it lb}.
  \item[\kw{function}  getElement\/\LB{}lb \CO{} \tn{Integer}, length\CO{} \tn{Integer}\/\RB{} \returns{} \/\LB{}aNewVectorType\/\RB{}]~\\
    The same as getSlice, but is able to use the subscript notation.
\end{desc}

\label{builtin VectorOfInt}
VectorOfInt is Vector.of[Integer].

\label{builtin VectorOfChar}
VectorOfChar is Vector.of[Character].

\subsection{ImmutableVector, ImmutableVectorOfAny, ImmutableVectorOfInt, ImmutableVectorOfString}
\label{builtin ImmutableVector}

ImmutableVector is the name of an object that creates immutable vectors.
The object
ImmutableVector is immutable and has the following interface:

\begin{desc}
  \item[\kw{function} of\/\LB{}T \CO{} \tn{Type}\/\RB{} \returns{} \/\LB{}aNewVectorCreatorType\/\RB{}]\kw{forall} T
\end{desc}

\noindent The object resulting from
{\it \tn{ImmutableVector}.of\/\LB{}T\/\RB{}}
is a type and a creator with the following interface, named
aNewVectorCreatorType:

\begin{desc}
  \item[\kw{operation} create\/\LB{}length \CO{} \tn{Integer}\/\RB{} \returns{} \/\LB{}aNewVectorType\/\RB{}]~\\
    Create a new immutable vector of length {\it length}, all of whose
    elements are \kw{nil}.  This is not terribly useful.
  \item[\kw{operation} literal\/\LB{}other \CO{} Sequence.of\/\LB{}T\/\RB{}\/\RB{} \returns{} \/\LB{}aNewVectorType\/\RB{}]~\\
    Create a new immutable vector, initialized with all of the elements of
    {\it other}.
  \item[\kw{operation} literal\/\LB{}other \CO{} Sequence.of\/\LB{}T\/\RB{}, length \CO{} \tn{Integer}\/\RB{} \returns{} \/\LB{}aNewVectorType\/\RB{}]~\\
    Create a new immutable vector of length {\it length}, initialized with
    the first length elements of {\it other}.
  \item[\kw{operation} literal\/\LB{}other \CO{} Sequence.of\/\LB{}T\/\RB{}, start \CO{} Integer, length \CO{} \tn{Integer}\/\RB{} \returns{} \/\LB{}aNewVectorType\/\RB{}]~\\
    Create a new immutable vector of length {\it length}, initialized with
    the length elements of {\it other}, starting at index {\it start}.
\end{desc}

\noindent Objects with type {\it \tn{ImmutableVector}.of\/\LB{}T\/\RB{}} have the
following interface, called aNewVectorType:

\begin{desc}
  \item[\kw{function}  getElement\/\LB{}index \CO{} \tn{Integer}\/\RB{} \returns{} \/\LB{}T\/\RB{}]~\\
    Return the element at {\it index}.
  \item[\kw{function}  lowerbound \returns{} \/\LB{}\tn{Integer}\/\RB{}]~\\
    Return 0.
  \item[\kw{function}  upperbound \returns{} \/\LB{}\tn{Integer}\/\RB{}]~\\
    Return the largest valid index.
  \item[\kw{operation} catenate \/\LB{}other \CO{} aNewVectorType\/\RB{} \returns{} \LB{}aNewVectorType\/\RB{}]~\\
    Return a new Immutable vector containing all of the elements of self
    followed by all of the elements of other.
  \item[\kw{function}  getSlice\/\LB{}lb \CO{} \tn{Integer}, length\CO{} \tn{Integer}\/\RB{} \returns{} \/\LB{}aNewVectorType\/\RB{}]~\\
    Return a new object with type aNewVectorType containing the {\it length}
    elements starting a position {\it lb}.
  \item[\kw{function}  getElement\/\LB{}lb \CO{} \tn{Integer}, length\CO{} \tn{Integer}\/\RB{} \returns{} \/\LB{}aNewVectorType\/\RB{}]~\\
    The same as getSlice, but is able to use the subscript notation.
\end{desc}

\label{builtin ImmutableVectorOfAny}
ImmutableVectorOfAny is ImmutableVector.of{Any}.

\label{builtin ImmutableVectorOfInt}
ImmutableVectorOfInt is ImmutableVector.of{Integer}.

\label{builtin ImmutableVectorOfString}
ImmutableVectorOfString is ImmutableVector.of{String}.

