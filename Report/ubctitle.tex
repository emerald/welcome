\begin{titlepage}
\vspace*{0.16in}
\begin{center}
%\flushright{\fbox{\large {\bf Limited Circulation}}}\\[0.9in]
\Large
{\bf The Emerald Programming Language\footnote{This work was 
supported in part
by the Natural Science and Engineering Research Council of Canada under
Grants No. ??????,
by the National Science Foundation under Grants No. 
MCS-8004111 and DCR-8420945, by
K{\o}benhavns Universitet (the University of Copenhagen), Denmark under
Grant J.nr.\ 574-2,2, by Digital Equipment Corporation
External Research Grants, and by an IBM Graduate Fellowship.}} \\[0.1in]
\large
{\bf Report} \\
\large
\vspace{2.5ex}
Norman C. Hutchinson, Rajendra K. Raj, \\
Andrew P. Black, Henry M. Levy, and Eric Jul\footnote{Authors' current 
addresses:  
Norman Hutchinson, Department of Computer Science, University of
British Columbia, Vancouver, BC, Canada V6T 1Z4.  
Andrew Black, Digital Equipment Corporation, 550 King Street, Littleton, MA 01460.
Eric Jul, DIKU, Department of Computer Science, University of Copenhagen, 
Universitetsparken 1, DK-2100 Copenhagen, Denmark.} \\[0.5in]
\normalsize
Technical Report 91-??\\
October 1991\\[0.5in]
{\em Department of Computer Science \\
University of British Columbia \\
Vancouver BC, Canada V6T 1Z2} \\
\vspace{2.5ex}
\end{center}
\vspace*{0.4in}

\begin{center}
\parbox{6.0in}{
{\bf Abstract:}\\[0.15in]
The programming language Emerald was designed and developed to demonstrate
that the object-based style of programming can be incorporated both
elegantly and efficiently in a distributed programming environment.  At
the same time, Emerald is a modern programming language providing excellent
features for abstractions and polymorphism.  Primarily a language for
distributed environments, Emerald includes features for dealing with the
location of objects, and extends exception handling mechanisms to
recovering from partial failures of distributed systems.  This report
presents a concise description of the Emerald programming language. 
}
\end{center}
\end{titlepage}
